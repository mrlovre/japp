\documentclass[a5paper,10pt]{tekst}

\begin{document}
	\thispagestyle{empty}
	\onehalfspacing
%	\vspace*{2em}
	\begin{center}
		\Huge\bfseries NASAの探査機が火星に着いた\\生命がいたか調べる
	\end{center}
	\vspace{2em}
	\begin{flushright}
		\Large 2021年2月19日
	\end{flushright}
	\vspace{2em}
	
	\CJKspace
	{\Large\sloppy
		\mbox{アメリカの} \mbox{NASAは} \mbox{去年} \mbox{7月に}、\mbox{火星を} \mbox{調べる} \mbox{ために} 「\mbox{パーシビアランス}」 \mbox{という} \mbox{探査機を} \mbox{打ち上げました}。
		\mbox{7か月で} \mbox{約4億7000万km} \mbox{飛んで}、 \mbox{日本の} \mbox{時間の} \mbox{19日朝}、 \mbox{火星に} \mbox{下りました}。
		
		\mbox{探査機が} \mbox{下りた} \mbox{所は}、 \mbox{昔は} \mbox{湖だったと} \mbox{考えられて} \mbox{いる} \mbox{所で}、 \mbox{生命が} \mbox{いたと} \mbox{わかる} \mbox{ものが} \mbox{残っている} \mbox{かも} \mbox{しれません}。
		
		\mbox{NASAは} \mbox{探査機に} \mbox{重さ} \mbox{2kgの} \mbox{小さい} \mbox{ヘリコプターを} \mbox{乗せました。}
		\mbox{ヘリコプターが} \mbox{火星の} \mbox{薄い} \mbox{大気の} \mbox{中で} \mbox{飛ぶことが} \mbox{できるか} \mbox{どうか} \mbox{調べる} \mbox{予定} \mbox{です}。
		\mbox{火星の} \mbox{土} \mbox{などを} \mbox{地球に} \mbox{持って} \mbox{帰る} \mbox{計画も} \mbox{あります}。
		
		\mbox{2年} \mbox{ぐらい} \mbox{火星を} \mbox{調べる} \mbox{予定で}、 \mbox{どの} \mbox{ように} \mbox{生命が} \mbox{生まれたか} \mbox{などが} \mbox{わかるかも} \mbox{しれません}。
		
	}

	\clearpage
	
	\subsection*{Vokabular}
	\renewcommand{\arraystretch}{1.333}
	\begin{xltabular}{\linewidth}{>{\Large}l>{\Large}ll>{\raggedright\arraybackslash}X}
		\toprule
		\textbf{漢字} & \textbf{ひらがな} & \textbf{značenje} & \textbf{vrsta riječi} \\
		\midrule
\endhead
		火星 & かせい & Mars & imenica \\
		調べる & しらべる & istražiti & glagol (一) \\
		探す & さがす & tražiti & glagol (五) \\
		探査機 & たんさき & sonda & imenica \\
		湖 & みずうみ & jezero & imenica \\
		生命 & せいめい & život (biološki) & imenica \\
		残る & のこる & ostati / zaostati & glagol (五) \\
		大気 & たいき & atmosfera & imenica \\
		予定 & よてい & plan & imenica, suru glagol \\
		着く & つく & stići / sjediti za (stolom) & glagol (五) \\
		去年 & きょねん & prošla godina & vremenska / priložna imenica \\
		月 & つき / げつ & mjesec & imenica \\
		打ち上げる & うちあげる & lansirati & glagol (一) \\
		約 & やく & približno & prilog \\
		億 & おく & tisuću milijuna (10\textsuperscript{8}) & broj \\
		万 & まん & deset tisuća (10\textsuperscript{4}) & broj \\
		飛ぶ & とぶ & letjeti & glagol (五) \\
		時間 & じかん & vrijeme & imenica \\
		朝 & あさ & jutro & imenica \\
		下りる & おりる & spustiti se / sletjeti & glagol (一) \\
		所 & ところ & mjesto & imenica, sufiks \\
		昔 & むかし & prije & no-pridjev, vremenska / priložna imenica \\
		考える & かんがえる & misliti & glagol (一) \\
		重さ & おもさ & težina & imenica \\
		小さい & ちいさい & malen & i-pridjev \\
		乗せる & のせる & staviti (na nešto) & glagol (一) \\
		薄い & うすい & rijedak & i-pridjev \\
		土 & つち & zemlja & imenica \\
		地球 & ちきゅう & Zemlja (planet) & imenica \\
		持って帰る & もってかえる & donijeti natrag & izraz \\
		計画 & けいかく & plan & imenica, suru glagol \\
		生まれる & うまれる & roditi se & glagol (一) \\ 
		\bottomrule
	\end{xltabular}

	\clearpage
	
	\subsection*{Zadaci}
	\begin{enumerate}
		\item NASAは去年何をした?
		\item 探査機はいつ火星についた?
		\item NASAは火星について何を調べていたんですか?
		\item 火星の状況を説明してください。
		\item Napišite kratku priču ili par rečenica koristeći barem 5 riječi iz teksta. 
		\item Sažmite tekst u najviše dvije rečenice.
	\end{enumerate}
\end{document}