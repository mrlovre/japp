%!TeX program = xelatex
\documentclass[10pt]{amsart}

\usepackage{multicol}
\usepackage{setspace}
\usepackage{titlesec}
\usepackage{indentfirst}
\usepackage{tikz}
\newcommand{\m}[1]{\kern -0.4em{#1}}
\usepackage{fontspec}
\usepackage{zxjatype}
\usepackage[a5paper, margin=2cm, includeheadfoot=true]{geometry}

\usepackage{enumitem}
\setlist[itemize]{wide, itemsep=0.0em, topsep=0.0em, parsep=0.0em, labelwidth=!, labelindent=0.5ex, leftmargin=1ex}
\setlist[enumerate]{wide, itemsep=0.0em, topsep=0.0em, parsep=0.0em, labelwidth=!, labelindent=0.5ex, leftmargin=1ex}

\setlength{\columnsep}{2em}

\setCJKmainfont[Scale=MatchUppercase, BoldFont={* SemiBold}, ItalicFont=*]{Source Han Serif JP}
\setCJKsansfont[Scale=MatchUppercase, BoldFont={* Bold}, ItalicFont=*]{Source Han Sans JP}
\setmainfont[BoldFont={* Semibold}]{Source Serif Pro}

\pgfdeclarelayer{bg}
\pgfsetlayers{bg,main}

\newcommand{\pitch}[1]{\raisebox{-\dp\strutbox}{
		\begin{tikzpicture}[
			every node/.style={inner sep=0.0, outer sep=0.0},
			dot/.style={draw, black, circle, anchor=center, minimum size=0.4ex, fill}
			]
			\foreach[count=\i from 1] \x/\p in {#1} 
			\ifx\x\empty
			\draw node (P\i) at (\i * 1em, 0) {} node (C\i) at (\i * 1em, \p * 0.4em + 0.8em) {}
			\else
			\draw node (P\i) at (\i * 1em, 0) {\x\strut} node[dot] (C\i) at (\i * 1em, \p * 0.4em + 0.8em) {}
			\fi;
			
			\begin{pgfonlayer}{bg} 
				\foreach[count=\i from 1, evaluate=\i as \j using int(\i - 1)] \x/\p in {#1}
				\ifnum \j>0
				\ifx\x\empty
				\draw[dash pattern=on 1.5pt off 1.5pt]({C\i}.center) -- ({C\j}.center)
				\else
				\draw[solid]({C\i}.center) -- ({C\j}.center)
				\fi
				\fi;
			\end{pgfonlayer}
	\end{tikzpicture}}
}

\newcommand{\dictentryadv}[4]{%
	\begin{minipage}{\linewidth}%
		\singlespacing%
		%		\raggedright%
		{\Large\vspace{-1em} #1 \hspace*{\fill}}%
		{\normalsize \mbox{\raggedleft #2}\hspace*{-1em}}\\
		{\small #4}
		\begin{itemize}#3\end{itemize}%
	\end{minipage}\vspace{1.5em}
}

\title{最後の授業}
\author{辞書}

\begin{document}
	\maketitle
	\thispagestyle{empty}
	\vspace{1em}
	
	\begin{multicols}{2}\noindent
		\dictentryadv{遅い \strut}{\pitch{お/0,そ/1,い/1,/1}}{\item slow \item late (e.g. "late at night") \item too late \item dull; stupid}{形容詞}
\dictentryadv{文法 \strut}{\pitch{ぶ/0,ん/1,ぽ/1,う/1,/1}}{\item grammar; syntax}{名詞}
\dictentryadv{質問 \strut}{\pitch{し/0,つ/1,も/1,ん/1,/1}}{\item question; inquiry; enquiry}{名詞}
\dictentryadv{叱る \strut}{\pitch{し/0,か/1,る/1,/1}}{\item to scold; to chide; to rebuke; to reprimand}{動詞}
\dictentryadv{怖い \strut}{\pitch{こ/0,わ/1,い/0,/0}}{\item scary; frightening; eerie; dreadful \item (I'm) afraid}{形容詞}
\dictentryadv{休む \strut}{\pitch{や/0,す/1,む/0,/0}}{\item to be absent; to take a day off \item to rest; to have a break \item to go to bed; to (lie down to) sleep; to turn in; to retire \item to stop doing some ongoing activity for a time; to suspend business}{動詞}
\dictentryadv{遊ぶ \strut}{\pitch{あ/0,そ/1,ぶ/1,/1}}{\item to play; to enjoy oneself; to have a good time \item to mess about (with alcohol, gambling, philandery, etc.) \item to be idle; to do nothing; to be unused \item to go to (for pleasure or for study) \item to intentionally throw a ball to lower the batter's concentration}{動詞}
\dictentryadv{考える \strut}{\pitch{か/0,ん/1,が/1,え/0,る/0,/0}}{\item to think about; to take into consideration \item to ponder; to reflect; to try to come at a conclusion; to think over (something) \item to intend; to decide (to do); to plan \item to come up with; to devise; to scheme \item to predict; to anticipate; to expect \item to suspect; to doubt \item to consider (somebody to be something); to look on}{動詞}
\dictentryadv{空 \strut}{\pitch{そ/1,ら/0,/0}}{\item emptiness; vacuum; blank}{名詞}
\dictentryadv{晴れる \strut}{\pitch{は/0,れ/1,る/0,/0}}{\item to clear up; to clear away; to be sunny; to stop raining \item to refresh (e.g. spirits) \item to be cleared (e.g. of a suspicion) \item to be dispelled; to be banished}{動詞}
\dictentryadv{暖か \strut}{\pitch{あ/0,た/1,た/1,か/0,/0}}{\item warm; mild; genial}{形状詞}
\dictentryadv{森 \strut}{\pitch{も/0,り/1,/1}}{\item forest \item shrine grove}{名詞}
\dictentryadv{鶫 \strut}{\pitch{つ/0,ぐ/1,み/1,/1}}{\item thrush (esp. the dusky thrush, Turdus naumanni)}{名詞}
\dictentryadv{鳴く \strut}{\pitch{な/0,く/1,/1}}{\item to sing (bird) \item to make sound (animal); to call; to cry; to chirp \item to make a meld call (e.g. pung, kong)}{動詞}
\dictentryadv{原っぱ \strut}{\pitch{は/1,ら/0,っ/0,ぱ/0,/0}}{\item open field; empty lot; plain}{名詞}
\dictentryadv{木挽き \strut}{\pitch{こ/0,び/1,き/1,/1}}{\item sawyer}{名詞}
\dictentryadv{工場 \strut}{\pitch{こ/0,う/1,じ/1,ょ/0,う/0,/0}}{\item factory; plant; mill; workshop}{名詞}
\dictentryadv{後ろ \strut}{\pitch{う/0,し/1,ろ/1,/1}}{\item back; behind; rear}{名詞}
\dictentryadv{兵隊 \strut}{\pitch{へ/0,い/1,た/1,い/1,/1}}{\item soldier; sailor}{名詞}
\dictentryadv{訓練 \strut}{\pitch{く/1,ん/0,れ/0,ん/0,/0}}{\item training; drill; practice; discipline}{名詞}
\dictentryadv{規則 \strut}{\pitch{き/1,そ/0,く/0,/0}}{\item rules; regulations; conventions}{名詞}
\dictentryadv{心 \strut}{\pitch{こ/0,こ/1,ろ/0,/0}}{\item mind; heart; spirit \item the meaning of a phrase (riddle, etc.)}{名詞}
\dictentryadv{引き付ける \strut}{\pitch{ひ/0,き/1,つ/1,け/1,る/0,/0}}{\item to fascinate; to attract; to charm \item to have a convulsion}{動詞}
\dictentryadv{誘惑 \strut}{\pitch{ゆ/0,う/1,わ/1,く/1,/1}}{\item temptation; allurement; lure; enticement; seduction}{名詞}
\dictentryadv{打ち勝つ \strut}{\pitch{う/0,ち/1,か/1,つ/0,/0}}{\item to conquer (e.g. an enemy); to defeat \item to overcome (a difficulty) \item to out-hit}{動詞}
\dictentryadv{急ぐ \strut}{\pitch{い/0,そ/1,ぐ/0,/0}}{\item to hurry; to rush; to hasten; to make something happen sooner}{動詞}
\dictentryadv{向かう \strut}{\pitch{む/0,か/1,う/1,/1}}{\item to face \item to go towards; to head towards}{動詞}
\dictentryadv{役場 \strut}{\pitch{や/0,く/1,ば/1,/0}}{\item town hall}{名詞}
\dictentryadv{側 \strut}{\pitch{そ/1,ば/0,/0}}{\item near; close; beside; vicinity; proximity; besides; while \item third person}{名詞}
\dictentryadv{通る \strut}{\pitch{と/1,お/0,る/0,/0}}{\item to go by; to go past; to go along; to travel along; to pass through; to use (a road); to take (a route); to go via; to go by way of \item to run (between); to operate (between); to connect \item to go indoors; to go into a room; to be admitted; to be shown in; to be ushered in; to come in \item to penetrate; to pierce; to skewer; to go through; to come through \item to permeate; to soak into; to spread throughout }{動詞}
\dictentryadv{金網 \strut}{\pitch{か/0,な/1,あ/1,み/1,/1}}{\item wire netting; wire mesh; wire screen; chain-link mesh}{名詞}
\dictentryadv{張る \strut}{\pitch{は/0,る/1,/1}}{\item to stick; to paste; to affix \item to stretch; to spread; to strain; to tighten; to put up (tent) \item to form (e.g. ice on a pond) \item to fill; to swell \item to stick out; to put; to slap \item to be expensive \item to keep a watch on; to be on the lookout \item to become one tile away from completion \item to span; to generate}{動詞}
\dictentryadv{掲示 \strut}{\pitch{け/0,い/1,じ/1,/1}}{\item notice; bulletin; post; posting; placard}{名詞}
\dictentryadv{先 \strut}{\pitch{ま/1,え/0,/0}}{\item previous; prior; former; first; earlier; some time ago; preceding \item point (e.g. pencil); tip; end; nozzle \item head (of a line); front \item ahead; the other side \item the future; hereafter \item destination \item the other party}{名詞}
\dictentryadv{大勢 \strut}{\pitch{お/0,お/1,ぜ/1,い/0,/0}}{\item crowd of people; great number of people \item in great numbers}{名詞}
\dictentryadv{立ち止まる \strut}{\pitch{た/0,ち/1,ど/1,ま/1,る/1,/1}}{\item to stop; to halt; to stand still}{動詞}
\dictentryadv{敗戦 \strut}{\pitch{は/0,い/1,せ/1,ん/1,/1}}{\item defeat; losing a war}{名詞}
\dictentryadv{挑発 \strut}{\pitch{ち/0,ょ/1,う/1,は/1,つ/1,/1}}{\item provocation; stirring up; arousal}{名詞}
\dictentryadv{司令 \strut}{\pitch{し/0,れ/1,い/1,/1}}{\item command; control; commander}{名詞}
\dictentryadv{部 \strut}{\pitch{ぶ/1,/0}}{\item department (in an organization); division; bureau \item club \item part; component; element \item category \item counter for copies of a newspaper or magazine}{名詞}
\dictentryadv{命令 \strut}{\pitch{め/0,い/1,れ/1,い/1,/1}}{\item order; command; decree; directive \item (software) instruction; statement}{名詞}
\dictentryadv{様 \strut}{\pitch{よ/1,う/0,/0}}{\item mess; sorry state; plight; sad sight \item -ways; -wards \item in the act of ...; just as one is ... \item manner of ...; way of ...}{形状詞}
\dictentryadv{嫌 \strut}{\pitch{い/0,や/1,/0}}{\item disagreeable; detestable; unpleasant; reluctant}{形状詞}
\dictentryadv{知らせ \strut}{\pitch{し/0,ら/1,せ/1,/1}}{\item news; word; tidings; notice; notification; information \item omen}{名詞}
\dictentryadv{皆 \strut}{\pitch{み/0,ん/1,な/1,/0}}{\item all; everyone; everybody \item everything}{名詞}
\dictentryadv{歩く \strut}{\pitch{あ/0,る/1,く/0,/0}}{\item to walk}{動詞}
\dictentryadv{今度 \strut}{\pitch{こ/1,ん/0,ど/0,/0}}{\item this time; now \item next time; another time; shortly; soon \item recently; previous; last}{名詞}
\dictentryadv{小走り \strut}{\pitch{こ/0,ば/1,し/0,り/0,/0}}{\item trot; half run; jog \item girl in charge of menial tasks in a samurai family}{名詞}
\dictentryadv{通り過ぎる \strut}{\pitch{と/0,お/1,り/1,す/1,ぎ/1,る/0,/0}}{\item to pass; to pass through}{動詞}
\dictentryadv{弟子 \strut}{\pitch{で/0,し/1,/0}}{\item pupil; disciple; adherent; follower; apprentice; young person; teacher's student-helper}{名詞}
\dictentryadv{一緒 \strut}{\pitch{い/0,っ/1,し/1,ょ/1,/1}}{\item together \item at the same time \item same; identical}{名詞}
\dictentryadv{読む \strut}{\pitch{よ/1,む/0,/0}}{\item to read \item to count \item to guess; to predict; to read (someone's thoughts); to see (e.g. into someone's heart); to divine}{動詞}
\dictentryadv{鍛冶屋 \strut}{\pitch{か/0,じ/1,や/1,/1}}{\item smith; blacksmith}{名詞}
\dictentryadv{大声 \strut}{\pitch{お/0,お/1,ご/1,え/0,/0}}{\item loud voice}{名詞}
\dictentryadv{坊や \strut}{\pitch{ぼ/1,う/0,や/0,/0}}{\item boy}{名詞}
\dictentryadv{良い \strut}{\pitch{い/1,い/0,/0}}{\item good; excellent; fine; nice; pleasant; agreeable \item sufficient (can be used to turn down an offer); ready; prepared \item profitable (e.g. deal, business offer, etc.); beneficial \item OK}{形容詞}
\dictentryadv{無い \strut}{\pitch{な/1,い/0,/0}}{\item nonexistent; not being (there) \item unpossessed; unowned; not had \item unique \item indicates negation, inexperience, unnecessariness or impossibility \item not ... \item to not be...; to have not ...}{形容詞}
\dictentryadv{伯父 \strut}{\pitch{お/0,じ/1,/1}}{\item uncle}{名詞}
\dictentryadv{思う \strut}{\pitch{お/0,も/1,う/0,/0}}{\item to think; to consider; to believe \item to think (of doing); to plan (to do) \item to judge; to assess; to regard \item to imagine; to suppose; to dream \item to expect; to look forward to \item to feel; to desire; to want \item to recall; to remember}{動詞}
\dictentryadv{息 \strut}{\pitch{い/1,き/0,/0}}{\item breath; breathing \item tone; mood}{名詞}
\dictentryadv{弾む \strut}{\pitch{は/0,ず/1,む/1,/1}}{\item to spring; to bound; to bounce; to be stimulated; to be encouraged; to get lively; to treat oneself to; to splurge on}{動詞}
\dictentryadv{間 \strut}{\pitch{あ/0,い/1,だ/1,/1}}{\item space (between); gap; interval; distance \item time (between); pause; break \item span (temporal or spatial); stretch; period (while) \item relationship (between, among) \item members (within, among) \item due to; because of}{名詞}
\dictentryadv{潜る \strut}{\pitch{く/0,ぐ/1,る/0,/0}}{\item to go under; to pass under; to go through; to pass through \item to dive (into or under the water) \item to evade; to get around; to slip past \item to survive; to surmount}{動詞}
\dictentryadv{授業 \strut}{\pitch{じ/1,ゅ/0,ぎ/0,ょ/0,う/0,/0}}{\item lesson; class work; teaching; instruction}{名詞}
\dictentryadv{始まり \strut}{\pitch{は/0,じ/1,ま/1,り/1,/1}}{\item origin; beginning}{名詞}
\dictentryadv{大変 \strut}{\pitch{た/0,い/1,へ/1,ん/1,/1}}{\item very; greatly; terribly; awfully \item immense; enormous; great \item serious; grave; dreadful; terrible \item difficult; hard \item major incident; disaster}{形状詞}
\dictentryadv{騒ぎ \strut}{\pitch{さ/1,わ/0,ぎ/0,/0}}{\item uproar; disturbance}{名詞}
\dictentryadv{机 \strut}{\pitch{つ/0,く/1,え/1,/1}}{\item desk}{名詞}
\dictentryadv{開ける \strut}{\pitch{あ/0,け/1,る/1,/1}}{\item to open (a door, etc.); to unwrap (e.g. parcel, package); to unlock \item to open (for business, etc.) \item to empty; to remove; to make space; to make room \item to move out; to clear out \item to be away from (e.g. one's house); to leave (temporarily) \item to dawn; to grow light \item to end (of a period, season) \item to begin (of the New Year) \item to leave (one's schedule) open; to make time (for) \item to make (a hole); to open up (a hole)}{動詞}
\dictentryadv{締める \strut}{\pitch{し/0,め/1,る/0,/0}}{\item to tie; to fasten; to tighten \item to wear (necktie, belt); to put on \item to total; to sum \item to be strict with \item to economize; to economise; to cut down on \item to salt; to marinate; to pickle; to make sushi adding a mixture of vinegar and salt}{動詞}
\dictentryadv{音 \strut}{\pitch{お/0,と/1,/0}}{\item sound; noise; report \item note \item fame \item Chinese-derived character reading}{名詞}
\dictentryadv{日課 \strut}{\pitch{に/0,っ/1,か/1,/1}}{\item daily lesson; daily work; daily routine}{名詞}
\dictentryadv{暗記 \strut}{\pitch{あ/0,ん/1,き/1,/1}}{\item memorization; memorisation; learning by heart}{名詞}
\dictentryadv{耳 \strut}{\pitch{み/0,み/1,/0}}{\item ear \item hearing \item edge; crust \item selvedge (non-fray machined edge of fabrics); selvage}{名詞}
\dictentryadv{手 \strut}{\pitch{て/1,/0}}{\item hand; arm \item forepaw; foreleg \item handle \item hand; worker; help \item trouble; care; effort \item means; way; trick; move; technique; workmanship \item hand; handwriting \item kind; type; sort \item one's hands; one's possession \item ability to cope \item hand (of cards) \item direction \item move (in go, shogi, etc.)}{名詞}
\dictentryadv{塞ぐ \strut}{\pitch{ふ/0,さ/1,ぐ/1,/1}}{\item to stop up; to close up; to block (up); to plug up; to shut up; to cover (ears, eyes, etc.); to close (eyes, mouth) \item to stand in the way; to obstruct \item to occupy; to fill up; to take up \item to perform one's role; to do one's duty \item to feel depressed; to mope}{動詞}
\dictentryadv{繰り返す \strut}{\pitch{く/0,り/1,か/1,え/1,す/1,/1}}{\item to repeat; to do something over again}{動詞}
\dictentryadv{声 \strut}{\pitch{こ/1,え/0,/0}}{\item voice}{名詞}
\dictentryadv{静か \strut}{\pitch{し/1,ず/0,か/0,/0}}{\item quiet; silent \item slow; unhurried \item calm; peaceful}{形状詞}
\dictentryadv{定規 \strut}{\pitch{じ/1,ょ/0,う/0,ぎ/0,/0}}{\item (measuring) ruler}{名詞}
\dictentryadv{叩く \strut}{\pitch{た/0,た/1,く/0,/0}}{\item to strike; to clap; to knock; to beat; to tap; to pat \item to play drums \item to abuse; to flame (e.g. on the Internet); to insult \item to consult; to sound out \item to brag; to talk big \item to call; to invoke (e.g. a function)}{動詞}
\dictentryadv{叫ぶ \strut}{\pitch{さ/0,け/1,ぶ/0,/0}}{\item to shout; to cry; to scream; to shriek; to yell; to exclaim \item to clamor (for or against); to clamour (for or against)}{動詞}
\dictentryadv{往来 \strut}{\pitch{お/0,う/1,ら/1,い/1,/1}}{\item coming and going; traffic \item road; street \item association; socializing; socialising; fellowship; mutual visits \item recurring (e.g. thoughts) \item correspondence}{名詞}
\dictentryadv{騒ぐ \strut}{\pitch{さ/0,わ/1,ぐ/0,/0}}{\item to make noise; to make racket; to be noisy \item to rustle; to swoosh \item to make merry \item to clamor; to clamour; to make a fuss; to kick up a fuss \item to lose one's cool; to panic; to act flustered \item to feel tense; to be uneasy; to be excited}{動詞}
\dictentryadv{気づく \strut}{\pitch{き/0,づ/1,く/0,/0}}{\item to notice; to recognize; to recognise; to become aware of; to perceive; to realize; to realise}{動詞}
\dictentryadv{自分 \strut}{\pitch{じ/0,ぶ/1,ん/1,/1}}{\item myself; yourself; oneself; himself; herself \item I; me \item you}{名詞}
\dictentryadv{席 \strut}{\pitch{せ/1,き/0,/0}}{\item seat \item location (of a gathering, etc.); place \item position; post}{名詞}
\dictentryadv{所 \strut}{\pitch{と/0,こ/1,ろ/1,/0}}{\item place; spot; scene; site \item address \item district; area; locality \item one's house \item point; aspect; side; facet \item passage (in text); part \item space; room \item thing; matter \item whereupon; as a result \item about to; on the verge of \item was just doing; was in the process of doing; have just done; just finished doing}{名詞}
\dictentryadv{日 \strut}{\pitch{ひ/0,/1}}{\item day; days \item sun; sunshine; sunlight \item case (esp. unfortunate); event}{名詞}
\dictentryadv{日曜 \strut}{\pitch{に/0,ち/1,よ/1,う/1,/1}}{\item Sunday}{名詞}
\dictentryadv{空く \strut}{\pitch{あ/0,く/1,/1}}{\item to become less crowded; to thin out; to get empty \item to be hungry}{動詞}
\dictentryadv{窓越し \strut}{\pitch{ま/0,ど/1,ご/1,し/1,/1}}{\item viewing through a window; passing through a window; going through a window; doing through a window}{名詞}
\dictentryadv{者 \strut}{\pitch{も/0,の/1,/0}}{\item person}{名詞}
\dictentryadv{恐ろしい \strut}{\pitch{お/0,そ/1,ろ/1,し/1,い/0,/0}}{\item terrible; dreadful; terrifying; frightening; frightened \item surprising; startling; tremendous; amazing}{形容詞}
\dictentryadv{抱える \strut}{\pitch{か/0,か/1,え/1,る/1,/1}}{\item to hold or carry under or in the arms \item to have (esp. problems, debts, etc.) \item to employ; to engage; to hire}{動詞}
\dictentryadv{来る \strut}{\pitch{く/1,る/0,/0}}{\item to come (spatially or temporally); to approach; to arrive \item to come back; to do ... and come back \item to come to be; to become; to get; to grow; to continue \item to come from; to be caused by; to derive from \item to come to (i.e. "when it comes to spinach ...")}{動詞}
\dictentryadv{ \strut}{\pitch{い/0,ら/1,っ/1,し/1,ゃ/0,る/0,/0}}{\item to come; to go; to be (somewhere) \item is (doing); are (doing)}{動詞}
\dictentryadv{戸 \strut}{\pitch{ど/0,/1}}{\item door (esp. Japanese-style) \item shutter; window shutter \item entrance (to a home) \item narrows}{名詞}
\dictentryadv{静まり返る \strut}{\pitch{し/0,ず/1,ま/1,り/1,か/1,え/0,る/0,/0}}{\item to fall silent; to become still as death}{動詞}
\dictentryadv{直中 \strut}{\pitch{た/0,だ/1,な/0,か/0,/0}}{\item middle}{名詞}
\dictentryadv{入る \strut}{\pitch{は/1,い/0,る/0,/0}}{\item to get in; to go in; to come in; to flow into; to set; to set in}{動詞}
\dictentryadv{恥ずかしい \strut}{\pitch{は/0,ず/1,か/1,し/1,い/0,/0}}{\item shy; ashamed; embarrassed \item disgraceful; shameful}{形容詞}
\dictentryadv{気 \strut}{\pitch{き/0,/1}}{\item spirit; mind; heart \item nature; disposition \item motivation; intention \item mood; feelings \item ambience; atmosphere; mood}{名詞}
\dictentryadv{大違い \strut}{\pitch{お/0,お/1,ち/1,が/0,い/0,/0}}{\item great difference \item major mistake}{形状詞}
\dictentryadv{起こる \strut}{\pitch{お/0,こ/1,る/0,/0}}{\item to occur; to happen}{動詞}
\dictentryadv{優しい \strut}{\pitch{や/0,さ/1,し/1,い/1,/1}}{\item tender; kind; gentle; graceful; affectionate; amiable}{形容詞}
\dictentryadv{口調 \strut}{\pitch{く/0,ち/1,ょ/1,う/1,/1}}{\item tone (e.g. of voice, etc.); (verbal) expression}{名詞}
\dictentryadv{早い \strut}{\pitch{は/0,や/1,い/0,/0}}{\item fast; quick; hasty; brisk \item early (in the day, etc.); premature \item (too) soon; not yet; (too) early \item easy; simple; quick}{形容詞}
\dictentryadv{為さる \strut}{\pitch{な/0,さ/1,る/0,/0}}{\item to do}{動詞}
\dictentryadv{始める \strut}{\pitch{は/0,じ/1,め/1,る/1,/1}}{\item to start; to begin; to commence; to initiate; to originate \item to open (e.g. a store); to start up; to establish (business. etc.) \item to start ...; to begin to ...}{動詞}
	\end{multicols}

\end{document}