\documentclass[a5paper, 10pt]{tekst}

\usepackage{titlesec}
\usepackage{indentfirst}

\begin{document}
	\thispagestyle{empty}
	\onehalfspacing
	\titleformat*{\section}{\sffamily\bfseries}
	\sffamily
	
	\begin{center}
		\Huge \p{\f{壁}{かべ}の}\p{\f{穴}{あな}}、\p{\f{第}{だい}}\p{\f{一}{いち}\f{話}{わ}}
	\end{center}
	
	\begin{figure}[h]
		\centering
		\includegraphics[width=0.5\linewidth]{figures/ねずみ2.jpg}
	\end{figure}
			
	{\Large\sloppy
		\noindent\p{「\f{腹}{はら}}\p{\f{減}{へ}った~!」と、}\p{\f{僕}{ぼく}は}\p{\f{伸}{の}びを}\p{した。}\p{そして}\p{\f{布団}{ふとん}から}\p{\f{起}{お}き}\p{\f{上}{あ}がった。}\p{\f{壁}{かべ}の}\p{\f{穴}{あな}の}\p{中は、}\p{まだ}\p{\f{薄暗}{うすぐら}かった。}\p{\f{僕}{ぼく}の}\p{\f{隣}{となり}の}\p{\f{布団}{ふとん}で、}\p{\f{妹}{いもうと}は}\p{まだ}\p{ぐっすりと}\p{\f{眠}{ねむ}って}\p{いた。}\p{\f{父}{とう}ちゃんと}\p{\f{母}{かあ}ちゃんも、}\p{まだ}\p{\f{起}{お}きて}\p{いない}\p{ようだった。でも、}\p{\f{僕}{ぼく}の}\p{\f{お腹}{おなか}は}\p{\f{グー}{ぐう}っと}\p{\f{鳴}{な}って、}\p{「そろそろ、}\p{\f{朝}{あさ}ごはんを}\p{\f{探}{さが}しに}\p{\f{行}{い}く}\p{\f{時間}{じかん}だ」と}\p{\f{知}{し}らせて}\p{いた。}
		
		\noindent\p{\f{僕}{ぼく}\f{達}{たち}の}\p{\f{家}{いえ}は、}\p{\f{壁}{かべ}の}\p{\f{裏}{うら}の}\p{\f{狭}{せま}い}\p{\f{空洞}{くうどう}だ。}\p{\f{僕}{ぼく}は、}\p{その}\p{\f{壁}{かべ}の}\p{\f{小}{ちい}さな}\p{\f{穴}{あな}から、}\p{そっと}\p{\f{顔}{かお}を}\p{出した。}\p{そして、}\p{\f{右}{みぎ}と}\p{\f{左}{ひだり}を}\p{ゆっくりと}\p{\f{確認}{かくにん}}\p{した。}\p{\f{壁}{かべ}の}\p{\f{穴}{あな}の}\p{外では、}\p{\f{決}{けっ}して}\p{\f{音}{おと}を}\p{\f{立}{た}てては}\p{いけない。}\p{\f{目立}{めだ}つ}\p{\f{行動}{こうどう}も}\p{\f{禁止}{きんし}だ。}\p{そして、}\p{\f{壁}{かべ}の}\p{\f{穴}{あな}に}\p{\f{戻}{もど}って}\p{\f{来}{く}る}\p{時は、}\p{\f{決}{けっ}して}\p{\f{誰}{だれ}にも}\p{見られては}\p{いけない。}\p{これが、}\p{この}\p{\f{壁}{かべ}の}\p{\f{穴}{あな}の}\p{ルール。}\p{そう、}\p{\f{僕}{ぼく}の}\p{\f{家}{いえ}の}\p{\f{掟}{おきて}だ。}
		
		\noindent\p{ここに}\p{\f{引}{ひ}っ}\p{\f{越}{こ}して}\p{\f{来}{き}たのは、}\p{ちょうど}\p{3}\p{\f{か月}{かげつ}}\p{前だった。}\p{前に}\p{\f{住}{す}んで}\p{いた}\p{\f{穴}{あな}は、}\p{ここよりも}\p{かなり}\p{\f{狭}{せま}かった。}\p{\f{遊}{あそ}べる}\p{スペースなんて、}\p{\f{全}{まった}く}\p{なかった。}\p{そして、}\p{\f{妹}{いもうと}\f{達}{たち}が}\p{3\f{匹}{ぴき}も}\p{\f{生}{う}まれて、}\p{もっと}\p{\f{窮屈}{きゅうくつ}に}\p{なった。}\p{そんな}\p{時、}\p{\f{人間}{にんげん}に}\p{\f{見}{み}つかって}\p{しまったんだ。}
		
	}\clearpage
	\titleformat*{\section}{\rmfamily\bfseries}
	\rmfamily
	
	\section*{Vokabular}
	\begin{multicols}{2}[\centering 
		\textbf{Bitno}]\noindent
		\dictentry{眠る}{ねむる}{\item spavati}{glagol (五)}
		\dictentry{探す}{さがす}{\item tražiti}{glagol (五)}
		\dictentry{知らせる}{しらせる}{\item obavjestiti}{glagol (一)}
		\dictentry{顔を出す}{かおをだす}{\item pokazati se}{fraza, glagol (五)}
		\dictentry{確認}{かくにん}{\item potvrda}{imenica, suru-glagol}
		\dictentry{決して}{けっして}{\item nikad}{prilog}
		\dictentry{音を立てる}{おとをたてる}{\item bučiti, proizvesti zvuk}{glagol (一)}
		\dictentry{目立つ}{めだつ}{\item isticati se, odstupati}{glagol (五)}
		\dictentry{行動}{こうどう}{\item pokret, ponašanje}{imenica, suru-glagol, no-pridjev}
		\dictentry{禁止}{きんし}{\item zabrana}{imenica, suru-glagol}
		\dictentry{全く}{まったく}{\item uistinu}{prilog}
		\dictentry{窮屈}{きゅうくつ}{\item skučen}{imenica, na-pridjev}
	\end{multicols}

	\clearpage
	\begin{multicols}{2}[\centering \textbf{Ostalo}]
		\dictentry{壁}{かべ}{\item zid}{imenica}
		\dictentry{穴}{あな}{\item rupa}{imenica}
		\dictentry{第}{だい}{\item prefiks za tvorbu rednih brojeva}{prefiks}
		\dictentry{一話}{いちわ}{\item prva epizoda}{brojač}
		\dictentry{腹}{はら}{\item trbuh}{imenica}
		\dictentry{減る}{へる}{\item smanjiti se}{glagol (五)}
		\dictentry{腹が減る}{はらがへる}{\item postati gladan, kolokvijalno}{izraz, glagol (五)}
		\dictentry{僕}{ぼく}{\item ja, muški}{zamjenica}
		\dictentry{伸び}{のび}{\item istezanje}{imenica, suru-glagol}
		\dictentry{布団}{ふとん}{\item futon}{imenica}
		\dictentry{起き上がる}{おきあがる}{\item ustati}{glagol (五)}
		\dictentry{中}{なか}{\item unutra}{imenica, no-pridjev, prilog}
		\dictentry{薄暗い}{うすぐらい}{\item mračan, sumoran}{i-pridjev}
		\dictentry{隣}{となり}{\item pored, susjedno, kuća do}{imenica}
		\dictentry{妹}{いもうと}{\item mlađa sestra}{imenica}
		\dictentry{父ちゃん}{とうちゃん}{\item tata}{imenica}
		\dictentry{母ちゃん}{かあちゃん}{\item mama}{imenica}
		\dictentry{起きる}{おきる}{\item ustati}{glagol (一)}
		\dictentry{お腹}{おなか}{\item trbuh}{imenica}
		\dictentry{鳴る}{なる}{\item zvoniti, odjekivati}{glagol (五)}
		\dictentry{朝ごはん}{あさごはん}{\item doručak}{imenica}
		\dictentry{行く}{いく}{\item ići}{glagol (五)}
		\dictentry{時間}{じかん}{\item vrijeme (period)}{imenica}
		\dictentry{僕達}{ぼくたち}{\item mi}{zamjenica}
		\dictentry{家}{いえ}{\item kuća}{imenica}
		\dictentry{裏}{うら}{\item dno}{imenica}
		\dictentry{狭い}{せまい}{\item uzak}{i-pridjev}
		\dictentry{空洞}{くうどう}{\item pukotina}{imenica}
		\dictentry{小さな}{ちいさな}{\item mala}{na-pridjev}
		\dictentry{右}{みぎ}{\item desno}{imenica}
		\dictentry{左}{ひだり}{\item lijevo}{imenica}
		\dictentry{外}{そと}{\item vani}{imenica}
		\dictentry{戻って来る}{もどってくる}{\item vratiti se}{fraza}
		\dictentry{時}{とき}{\item vrijeme}{imenica}
		\dictentry{誰にも}{だれにも}{\item nikome/svakome}{izraz}
		\dictentry{見る}{みる}{\item vidjeti}{glagol (一)}
		\dictentry{掟}{おきて}{\item zakon}{imenica}
		\dictentry{引っ越す}{ひっこす}{\item seliti se}{glagol (五)}
		\dictentry{来る}{くる}{\item doći}{nepravilan glagol}
		\dictentry{か月}{かげつ}{\item broj mjeseci}{brojač}
		\dictentry{前}{まえ}{\item prije}{priložna imenica}
		\dictentry{住む}{すむ}{\item stanovati}{glagol (五)}
		\dictentry{遊ぶ}{あそぶ}{\item igrati se}{glagol (五)}
		\dictentry{3匹}{さんびき}{\item 3 male životinje}{brojač}
		\dictentry{生まれる}{うまれる}{\item roditi se}{glagol (一)}
		\dictentry{人間}{にんげん}{\item čovjek}{imenica}
		\dictentry{見つかる}{みつかる}{\item biti nađen}{glagol (五)}	
	\end{multicols}
	
	\clearpage
	
	\section*{Domaća zadaća}
	\begin{enumerate}
		\item Napišite kratku priču ili par rečenica koristeći riječi iz kutije ispod.
		Rečenice ili tekst ne moraju nužno biti vezane uz samu vijest.
		\begin{center}
			\vspace{0.5em}
			\fbox{探す ・ 顔を出す ・ 確認 ・ 決して ・ 行動 ・ 全く}\vspace{1.4em}
			\rule{\linewidth}{0.5pt}\\[0.6em]
			\rule{\linewidth}{0.5pt}\\[0.6em]
			\rule{\linewidth}{0.5pt}\\[0.6em]
			\rule{\linewidth}{0.5pt}\\[0.6em]
			\rule{\linewidth}{0.5pt}\\[0.6em]
			\rule{\linewidth}{0.5pt}\\[0.6em]
			\rule{\linewidth}{0.5pt}
		\end{center}
		
		\item Odgovorite na pitanja:
		\begin{enumerate}[label=(\roman*)]
			\raggedright
			\item\vspace{0.5em}\p{\f{語}{かた}り}\p{\f{手}{て}は}\p{\f{誰}{だれ}だと}\p{\f{思}{おも}いますか?}
			\rule{\linewidth}{0.5pt}\\[0.6em]
			\rule{\linewidth}{0.5pt}
			\item\vspace{0.5em}\p{\f{壁}{かべ}の}\p{\f{穴}{あな}の}\p{中の}\p{\f{住人}{じゅうにん}は}\p{\f{何匹}{なんびき}}\p{いますか?}
			\rule{\linewidth}{0.5pt}\\[0.6em]
			\rule{\linewidth}{0.5pt}
			\item\vspace{0.5em}\p{\f{語}{かた}り}\p{\f{手}{て}は}\p{何を}\p{する}\p{\f{積}{つも}りですか?}
			\rule{\linewidth}{0.5pt}\\[0.6em]
			\rule{\linewidth}{0.5pt}
			\item\vspace{0.5em}\p{\f{語}{かた}り}\p{\f{手}{て}の}\p{\f{家}{いえ}の}\p{\f{掟}{おきて}は}\p{\f{何}{なん}ですか?}
			\rule{\linewidth}{0.5pt}\\[0.6em]
			\rule{\linewidth}{0.5pt}
			\item\vspace{0.5em}\p{いつ}\p{今の}\p{\f{穴}{あな}に}\p{\f{引}{ひ}っ}\p{\f{越}{こ}して}\p{きたのですか?}
			\rule{\linewidth}{0.5pt}\\[0.6em]
			\rule{\linewidth}{0.5pt}
			\item\vspace{0.5em}\p{前の}\p{\f{穴}{あな}は}\p{どんな}\p{\f{穴}{あな}でしたか?}
			\rule{\linewidth}{0.5pt}\\[0.6em]
			\rule{\linewidth}{0.5pt}
		\end{enumerate}		
		\item Nadopunite sljedeće rečenice riječima iz kutije ispod:
		\begin{center}
			\choicebox{眠っていた ・ 探している ・ 知らせました\\顔を出さない ・ 確認して ・ 決して ・ 音を立てた\\目立つ ・ 行動 ・ 禁止 ・ 全く ・ 窮屈}
		\end{center}
		\begin{enumerate}[label=(\roman*)]
			\raggedright
			
			\vspace{0.5em}\item \p{\f{花子}{はなご}ちゃん、}\p{\f{花子}{はなご}ちゃん、}\p{\f{昨日}{きのう}}\p{いつまで\ansline{}?}
			\item \p{\f{田中}{たなか}}\p{お}\p{じいちゃん、}\p{うちの}\p{\f{犬}{いぬ}を\ansline{}んだけど、}\p{\f{見}{み}かけましたか?}
			\item\p{\f{先生}{せんせい}に}\p{この}\p{\f{事}{こと}は\ansline{}か?}
			\item\p{たまには}\p{\f{店}{みせ}にも\ansline{}と}\p{\f{叱}{しか}られるのさ。}
			\item\p{\f{死体}{したい}を}\p{\f{棺桶}{かんおけ}に}\p{\f{入}{い}れる}\p{前は}\p{まだ}\p{\f{生}{い}きて}\p{いるか}\p{どうか\ansline{}}\p{ください。}
			\item\p{\f{私}{わたし}の}\p{ポテチには\ansline{}}\p{\f{手}{て}を}\p{出すな、}\p{\f{痛}{いた}い}\p{\f{目}{め}を}\p{見るよ。}
			\item\p{\f{小屋}{こや}の}\p{\f{屋根}{やね}は}\p{\f{雪}{ゆき}の}\p{\f{重}{おも}みで}\p{ミシミシと\ansline{}。}
			\item\p{\f{林}{はやし}に}\p{\f{竹}{たけ}が\ansline{}!}
			\item\p{\f{人間}{にんげん}は}\p{\f{狂}{くる}って}\p{いれば\ansline{}の}\p{\f{責任}{せきにん}は}\p{\f{問}{と}われない。}
			\item\p{\f{武}{たけし}\f{君}{くん}は}\p{\f{暗}{くら}い}\p{\f{道}{みち}を}\p{\f{歩}{ある}き}\p{ながら}\p{\f{変}{へん}な}\p{\f{音}{おと}を}\p{\f{聞}{き}いて}\p{\f{後}{うし}ろを}\p{\f{向}{む}いた、}\p{\f{築}{きず}けば}\p{\f{立}{た}{ち}}\p{\f{入}{い}り\ansline{}の}\p{サインに}\p{\f{囲}{かこ}まれて}\p{いた。}
			\item\p{ドラゴンマーテは}\p{ほかの}\p{ドラゴンから\ansline{}}\p{\f{違}{ちが}う}\p{\f{生活}{せいかつ}を}\p{して}\p{いた。}
			\item\p{\f{列車}{れっしゃ}の}\p{コンパートメントは}\p{すぐ\ansline{}に}\p{なる。}
		\end{enumerate}
	\end{enumerate}
\end{document}