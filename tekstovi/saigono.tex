%!TeX program = xelatex
\documentclass[12pt]{amsart}
\usepackage[a5paper, margin=2.2cm, includeheadfoot=true]{geometry}
\usepackage[croatian]{babel}
%\usepackage[PunctStyle=CCT,CJKchecksingle,CJKnumber,CJKspace=true]{xeCJK}
\usepackage{zxjatype}

\setCJKmainfont[Scale=MatchUppercase, BoldFont={* SemiBold}, ItalicFont=*]{Source Han Serif JP}
\setCJKsansfont[Scale=MatchUppercase, BoldFont={* Bold}, ItalicFont=*]{Source Han Sans JP}
\setmainfont[BoldFont={* Semibold}]{Source Serif Pro}

\usepackage{booktabs}
\usepackage{setspace}
\usepackage{enumitem}
\usepackage[defaultlines=4, all]{nowidow}

\usepackage{multicol}
\usepackage{pbox}

\usepackage[overlap,CJK]{ruby}
\renewcommand{\rubysep}{-3.7ex}
\renewcommand{\rubysize}{0.54}
\newcommand{\f}[2]{\ruby{#1}{\sffamily\mdseries\protect\furiganafix{#2}}\CJKglue}

\usepackage{xstring}
\newcommand{\squeeze}{\kern -0.2em}
\newcommand{\furiganafix}[1]{{%
		\StrSubstitute{#1}{・}{@・@}[\x]%
		\StrSubstitute{\x}{ゃ}{@ゃ@}[\x]%
		\StrSubstitute{\x}{ゅ}{@ゅ@}[\x]%
		\StrSubstitute{\x}{ょ}{@ょ@}[\x]%
		\StrSubstitute{\x}{ぁ}{@ぁ@}[\x]%
		\StrSubstitute{\x}{ぃ}{@ぃ@}[\x]%
		\StrSubstitute{\x}{ぅ}{@ぅ@}[\x]%
		\StrSubstitute{\x}{ぇ}{@ぇ@}[\x]%
		\StrSubstitute{\x}{ぉ}{@ぉ@}[\x]%
		\StrSubstitute{\x}{っ}{@っ@}[\x]%
		\StrSubstitute{\x}{ャ}{@ャ@}[\x]%
		\StrSubstitute{\x}{ュ}{@ュ@}[\x]%
		\StrSubstitute{\x}{ョ}{@ョ@}[\x]%
		\StrSubstitute{\x}{ァ}{@ァ@}[\x]%
		\StrSubstitute{\x}{ィ}{@ィ@}[\x]%
		\StrSubstitute{\x}{ゥ}{@ゥ@}[\x]%
		\StrSubstitute{\x}{ェ}{@ェ@}[\x]%
		\StrSubstitute{\x}{ォ}{@ォ@}[\x]%
		\StrSubstitute{\x}{ッ}{@ッ@}[\x]%
		\StrSubstitute{\x}{@@}{@}[\x]%
		\StrSubstitute{\x}{@}{\squeeze}[\x]%
		\x}}

\newcommand{\g}{\furiganafix}

\usepackage[babel=true]{microtype}
%\setlist[itemize]{wide, itemsep=0.0em, topsep=0.0em, parsep=0.0em, labelwidth=!, labelindent=0.5ex, leftmargin=1ex}
%\setlist[enumerate]{wide, itemsep=0.0em, topsep=0.0em, parsep=0.0em, labelwidth=!, labelindent=0.5ex, leftmargin=1ex}
\usepackage{makecell}

\title{最後の授業}
\author{アルフォーンズ・ドーデ}

\onehalfspacing

\begin{document}
	\maketitle
	\thispagestyle{empty}

	その朝は、学校へ行くのがたいへんおそくなったし、アメル先生から文法の\f{質問}{しつもん}をすると言われていたのに、わたしはなにも勉強していなかったので、しかられるのがこわかったのです。

	それで、学校を休んでどこかへ遊びにいこう、と考えました。

	空はよく晴れてあたたかでした。

	森のなかでは、つぐみが鳴いていまし、リベールの原っぱからは、\f{木}{こ}びき工場のうしろでプロシャの\f{兵隊}{へいたい}たちが訓練しているのがきこえます。
	森へいこうか、原っぱへいこうか、どれも、文法の\f{規則}{きそく}よりはわたしの心をひきつけました。
	けれど、やっとこのゆうわくにうち勝って、いそいで学校へむかってかけだしました。
	役場のそばをとおると、\f{金網}{かなあみ}を\f{張}{は}った小さな\f{掲示板}{けいじばん}の前に、おおぜいの人が立ちどまっていました。
	二年ほどまえから\f{敗戦}{はいせん}とか、\f{挑発}{ちょうはつ}とか、\f{司令部}{しれいぶ}の\f{命令}{めいれい}とかいうようないやなしらせは、みんな、ここにけ\f{掲示}{けいじ}されることになっていました。
	わたしは歩きながら考えました。

	〈こんどは、なんのしらせかしら?〉

	そして、小走りとおりすぎようとすると、そこで、\f{弟子}{でし}といっしょに\f{掲示}{けいじ}を読んでいた\f{か}{・}\f{じ}{・}\f{屋}{や}のワシュテルさんが、大声でわたしに言いました。

	「おい、ぼうや、そんなにいそがなくったっていいさ、どうせ学校にはおくれっこないんだから!」

	かじ\f{屋}{や}のおじさん、わたしをからかっているんだな、と思ったので、わたしは息をはずませて、学校の間をくぐりました。

	いつもなら、\f{授業}{じゅぎょう}のはじまりはたいへんなさわぎでした。
	つくえをばたばたあけたりしめたりする音や、日課を暗記しようと、耳を手でふさいで大声でくりかえしている声やら、「さ、すこし静かに!」と、じょうぎでつくえをたたきながら\f{叫}{さけ}ぶ先生の声が\f{往来}{おうらい}まできこえていたものでした。

	わたしは、みんながこうしてさわいでいれば、だれにも気づかれないで、そっと自分の席につくことができるだろうと思いました。ところがその日は、なにもかもひっそりとして、まるで、日曜の朝のようでした。あいている\f{窓}{まど}ごしになかを見ると、クラスの者はみんな自分の席についていますし、アメル先生が、あのおそろしいじょうぎをかかえて、いったりきたりしていらっしゃいます。
	戸をあけて、この静まりかえったまっただなかに入らなければならないことを思う、なんだかはずかしいような、こわいような気がします。

	ところが、大ちがいでした。
	アメル先生は、おこるどころか、わたしを見ると、やさしい\f{口調}{くちょう}で、こう言われました。
	「フランツか。早く席につきなさい。もうこないのかと思って、はじめるところだった。」

  わたしは、すぐに席につきました。
  そして、おそろしさがおさまると、わたしは、先生が\f{視学館}{しがくかん}のくる日とか、卒業式の日でなければ着ない、りっぱな緑色のフロックコート(\f{上着丈}{うわぎたけ}の長い、\f{男性用}{だんせいよう}の\f{礼服}{れいふく})を着て、こまかくひだをとった、はばのひろいネクタイをしめ、ししゅうをした、黒い\f{絹}{きぬ}のふしなし\f{帽}{ぼう}をかぶっていらっしゃるのに気がつきました。
  それに、教室全体に、なにかふしぎなおごそかさがみなぎっていました。

  いちばんおどろかされたのは、教室の後ろのほうの、いつもはあいている席とでした。
  \f{三角帽}{さんかくぼう}をもったオゼールじいさんや、もとのざいじょう村長さんや、\f{郵便屋}{ゆうびんや}さんの顔もみえます。
  そのほかにも、おおぜいの人がいましたが、みんな悲しそうでした。
  オゼールじいさんは、表紙のいたんだ古い\f{読本}{とくほん}をもってきていて、ひざの上にひろげ、大きなめがねをそのうえにおいていました。

  わたしがいろいろのことにびっくりしているまに、アメル先生は\f{教壇}{きょうだん}にあがって、わたしをむかえたときと同じような、やさしい重みのある声で話されました。

  「みなさん、わたしが\f{授業}{じゅぎょう}をするのは、これが最後になりました。
  アルゼスとロレーヌの学校では、ドイツ語しか教えてはいけないという命令が、ベルリンからきたのです。
  新しい先生が、明日、おみえになります。
  今日はフランス語の最後の\f{授業}{じゅぎょう}です。
  どうか、よく注意してきいてください。」

  わたしはびっくりしました。
  さっき役場に\f{掲示}{けいじ}してあったのは、このことだったのでしょう。

  ああ、フランス語の最後の\f{授業}{じゅぎょう}!

  それなのに、わたしはまだフランス語がやっと書けるくらいです。
  では、もう、習うことはできないのでしょうか。
  フランス語をもっと勉強することは、できなくなったのでしょうか。

  ああ、どうしてわたしは、いままで教室で、あんなにぼんやりしていたのだろう。
  鳥の\f{巣}{す}をさがしまわったり、氷すべりをするために学校をずるけたことを、自分ながらうらめしく思いました。
  さっきまで、あんなにじゃまだった文法の本や\f{聖書}{せいしょ}などが、いまでは、別れたくないむかしなじみの友だちのように思われました。
  アメル先生にたいしても、同じような気持ちを感じました。
  先生はどこかへいってしまうのだ、もう会うことはできないのだ、と思うと、先生にしかられたり、じょうぎで打たれたことも、わすれてしまいました。

  ああ、おきのどくな先生!

  先生は、この最後の\f{授業}{じゅぎょう}のために、着かざってこられたのでした。
  わたしは、なぜ村の\f{老人}{ろうじん}たちが、教室にきて後ろのほうにすわっているのかが、わかりました。
  どうやら、この学校にあまりたびたびこなかったことをくやんでいるようです。

  村の人たちは、また、先生の四十年ものあいだの苦労を\f{感謝}{かんしゃ}し、かえっていかれる\f{祖国}{そこく}にたいして\f{敬意}{けいい}をあらわすためにきたのでしょう……。

  わたしが、こうしてじいっと考えこんでいるとき、とつぜん、わたしの名まえが\f{呼}{よ}ばれました。
  わたしの\f{暗唱}{あんしょう}の番がきたのです。
  わたしは最初からまごついてしまって、立ったまま悲しい気持ちで、頭もあげられず、もじもじしていました。
  アメル先生の静かな声が、きこえてきました。

  「フランツ、わたしはしかりません。
  自分でよくわかるでしょう。
  『いま勉強しなくても、勉強するときはじゅうぶんある。
  あした勉強しよう』
  などというのが、わたしたちの口ぐせでしたね。
  そしてそのため、どうなったかおわりでしょう。
  今日勉強にのばす、これがアルゼスの大きな不幸だったのです。
  いま、ドイツ人たちに、こう言われてもしかたありません。
  『どうしたんだ、おまえたちはフランス人だと言いはっていた。
  それなのに、フランスの言葉を話すことも、書くことも、さっぱりできないじゃないか』。
  この点で、フランツ、あなたがいちばん悪いというわけではありません。
  わたしたちみんなが悪かったのです。
  みんなに\f{責任}{せきにん}があるのです。」

  アメル先生は、また続けられました。

  「あなたがたのおとうさんやおかあさんがたは、子どもたちが教育を受けることをあまりのぞまなかったのです。
  すこしでも\f{金}{かね}になれば、というわけで、畑や工場にいかせたがりました。
  いえ、こういうわたし自身にも、\f{責任}{せきにん}があります。
  勉強の時間に、あなたがたに花に水をやらせたこともあり、わたしがアユつりにいきたいために、あなたがたに休みをあたえたこともありました。」

  それからアメル先生は、フランス語についてつぎからつぎへと話をなさいました。
  フランス語が世界でいちばん美しい、いちばんはっきっりした、いちばん力強い言葉であることや、ある\f{民族}{みんぞく}がどれいとなっても、その国語をもっているうちは、その\f{牢獄}{ろうごく}のかぎをにぎっているようなものだから、わたしたちの間でフランス語をよく守りとおして、けっしてわすれないようにしなければならないというお話でしいた。

  それから、先生は、文法の本を開いて、今日のけいこのところをお読みになりました。
  わたしはあまりよくわかるので、びっくりしました。
  先生がおっしゃたことは、わたしには、たいへんやさしく思われました。
  わたしがこれほど注意してきいたことははじめてでしたし、先生がこれほどしんぼう強く説明されたことも、いままでありませんでした。
  先生は、この土地を去っていくまえに、知っていることをすっかり教えて、いっぺんにわたしたちの頭のなかへつめこまうとしていらっしゃるように思われました。

  本を用意しておいてくださいました。
  それには、まるみをおびた、きれいな字で、《フランス、アルゼス、フランス、アルゼス》と書いてありました。
  そのお手本はまるで、小さな旗がつくえのくぎにかかって、教室じゅうに、ひるがえっているように見えました。
  わたしたちは、いっしょうけんめいでした。
  みんな、しいんと静まりかえっています。
  ただ紙の上をペンの走る音がきこえるばかりです。
  とちゅうで一度\f{窓}{まど}からこがね虫が一ぴき入ってきましたが、そんなものに気をとられる者は、ひとりもいません。
  村の人といっしょに、おさない子どもまでが、一心に紙の上に線を引いていました。
  まるでその線のひとすじひとすじが、フランスの言葉であるかのように、まじめに、心をこめて書いているのです。
  学校の屋根の上では、ハトが静かに鳴いていました。
  わたしはその声を聞いて、〈今に、ハトまで、ドイツ語で鳴かなければならないのじゃないかしら?〉と思いました。

  ときどきページから目をあげて見ますと、アメル先生は\f{教壇}{きょうだん}の上に立って、あたりを静かにながめていらっしゃいます。
  まるで、小さな\f{校舎}{こうしゃ}をみんな目のなかにおさめようとしていらっしゃるようです。
  むりもありません。
  四十年もの長い間、ここで、すこしもかわらないこの教室で、教えてきたのですもの。
  ただかわったのは、つくえやこしかけが、使われている間に、こすられ、つやが出てきたぐらいものです。
  庭のクラミの木は大きくなり、先生の手植えのヒイラギが、いまは\f{窓}{まど}の外に美しくげって、屋根までとどくくらいになっています。
  こういうすべてのものと別れるということは、先生にとっては、どんなに悲しいことでしょう!
  二階では、先生の妹さんが\f{荷造}{にづく}りをしていらっしゃいますが、そのゆききする足音をきいて、先生は、きっと、\f{胸}{むね}のつぶれるような思いをされているでしょう。
  明日は、いよいよ出発です。
  \f{永遠}{えいえん}に、この土地を去らなければなっらないのです。

  それでも先生は勇気を出して、最後まで\f{授業}{じゅぎょう}を続けられました。
  習字のつぎは、歴史の勉強でした。
  それから小さな生徒たちは、みんないっしょに読み方のけいこをはじめました。
  \f{読本}{とくほん}を両手にもって、生徒たちといっしょに文字をひろい読みしていました。
  いっしょうけんめいなのがわかります。
  じいさんの声は、\f{感激}{かんげき}のわたしたちはみんな、笑いたくなり、泣きたくもなりました。

  ああ、この最後の\f{授業}{じゅぎょう}を、わたしは一生わすれることができません……。

  とつぜん、教室の時計が十二時を打ちました。
  つづいてアンジェリュスの\f{鐘}{かね}がきこえてきました。
  それと同時に、訓練からもどるプロシャ\f{兵隊}{へいたい}のラッパ
  が\f{窓}{まど}の外からひびいてきました。
  アメル先生は、すっと\f{教壇}{きょうだん}に立ちあげられました。
  頭は真っ青です。
  先生がこんなに大きく見えたことはありませんでした。

  「みんなさん」と、先生は言いました。

  「みんなさん……わたしは……わたしは……。」

\end{document}