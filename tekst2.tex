\documentclass[a5paper,10pt]{tekst}

%\setlength{\parindent}{2em}

\begin{document}
	\thispagestyle{empty}
	\onehalfspacing
	%	\vspace*{2em}
	\sffamily
	\begin{center}
		\Huge \f{都立}{とりつ} \f{高校}{こうこう}の 40% \f{髪}{かみ}が \f{茶色}{ちゃいろ}の \f{生徒}{せいと}に \f{証明書}{しょうめいしょ}を 出す ように 言う
	\end{center}
	\vspace{2em}
%	\begin{flushright}
%		\Large 2021年3月1日
%	\end{flushright}
%	\vspace{2em}
	
	{\Large\sloppy
	\p{東京の}\p{{都立}{高校}の}\p{40%}\p{\f{以上}{いじょう}が、}\p{{髪}の}\p{\f{毛}{け}が}\p{茶色や、}\p{まっすぐ}\p{では}\p{ない}\p{{生徒}に}\p{証明書を}\p{出す}\p{ように}\p{言って}\p{いる}\p{ことが}\p{わかりました。}
	\p{東京都}\p{\f{議会}{かいぎ}の}\p{\f{共産党}{きょうさんとう}の}\p{\f{議員}{ぎいん}が}\p{\f{調}{しら}べて}\p{わかりました。}
	
	\p{証明書}\p{では、}\p{生まれた}\p{ときから}\p{髪の}\p{毛が}\p{茶色や、}\p{まっすぐ}\p{では}\p{ない}\p{ことを}\p{書いて、}\p{\f{家族}{かぞく}が}\p{サイン}\p{などを}\p{します。}
	\p{小さい}\p{ときの}\p{\f{写真}{しゃしん}を}\p{出す}\p{ように}\p{言って}\p{いる}\p{\f{学校}{がっこう}も}\p{ありました。}
	
	\p{東京都の}\p{\f{教育}{きょういく}}\p{\f{委員会}{いいんかい}は}\p{「髪を}\p{\f{染}{そ}めたと}\p{先生が}\p{\f{間}{ま}\f{違}{ちが}えない}\p{ように}\p{出して}\p{もらって}\p{いますが、}\p{\f{必}{かなら}ず}\p{では}\p{ありません。}
	\p{生徒や}\p{家族}\p{などの}\p{\f{意見}{いけん}も}\p{聞いて、}\p{\f{毎年}{まいとし}、}\p{いい}\p{やり\f{方}{かた}を}\p{\f{考}{かんが}える}\p{\f{必要}{ひつよう}が}\p{あります」と}\p{説明}\p{して}\p{います。}
	\p{しかし、}\p{\f{専門家}{せんもんか}は}\p{「この}\p{やり方は}\p{\f{時代}{じだい}に}\p{\f{合}{あ}って}\p{いないし、}\p{子どもたちの}\p{\f{人権}{じんけん}の}\p{\f{問題}{もんだい}も}\p{あります」と}\p{\f{話}{はな}して}\p{います。}
		
	}
	
	\clearpage
	\rmfamily
	
	\subsection*{Vokabular}
	\renewcommand{\arraystretch}{1.333}
	\begin{xltabular}{\linewidth}{>{\normalsize}l>{\normalsize}l>{\raggedright\arraybackslash}l>{\raggedright\arraybackslash}X}
			\toprule
			\textbf{漢字} & \textbf{ひらがな} & \textbf{značenje} & \textbf{vrsta riječi} \\ \midrule
			\endhead
			\multicolumn{4}{c}{BITNO} \\ \midrule
			以上  & いじょう & više od & priložna imenica \\ 
			調べる & しらべる & istražiti & glagol  \\ 
			間違える & まちがえる & pogriješiti & glagol \\ 
			写真 & しゃしん & slika & imenica \\ 
			必ず & かならず & sigurno & prilog \\ 
			意見 & いけん & mišljenje & imenica \\ 
			考える & かんがえる & misliti & glagol \\ 
			必要 & ひつよう & potrebno & imenica, no-pridjev \\ 
			説明 & せつめい & objašnjenje & imenica \\ 
			問題 & もんだい & problem & imenica \\ 
			合う & あう & pristajati, biti prikladan & glagol \\ \midrule
			\multicolumn{4}{c}{KORISNO}  \\ \midrule
			髪 & かみ & kosa & imenica \\ 
			茶色  & ちゃいろ & smeđa & no-pridjev, imenica \\ 
			生徒 & せいと & učenik & imenica \\ 
			髪の毛 & かみのけ & kosa (dosl. dlaka na glavi) & imenica, izraz \\ 
			生まれる & うまれる & roditi se & glagol \\ 
			書く & かく & pisati & glagol \\ 
			家族 & かぞく & obitelj & imenica \\ 
			小さい & ちいさい & malen & i-pridjev \\ 
			学校 & がっこう & škola & imenica \\ 
			先生 & せんせい & profesor & imenica \\ 
			聞く & きく & čuti & glagol \\ 
			毎年 & まいとし & svake godine & vremenska imenica \\ 
			やり方 & やりかた & način rađenja & imenica \\ 
			子供 & こども & dijete & imenica \\ 
			話す & はなす & pričati & glagol \\ 
			時代 & じだい & doba & vremenska imenica \\ \midrule
			\multicolumn{4}{c}{OSTALO} \\ \midrule
			専門家 & せんもんか & stručnjak & imenica \\
			証明書  & しょうめいしょ & potvrda & imenica \\ 
			{東京都議会}  & \makecell[lt]{とうきょうと〜\\ ぎかい} & Tokyo gradska skupština & imenica \\ 
			都立高校 & とりつこうこう & gradska gimnazija & imenica \\ 
			共産党 & きょうさんとう & komunistička partija & imenica \\ 
			議員 & ぎいん & član stranke, odbora & imenica \\ 
			東京都 & とうきょうと & prefektura Tokyo & imenica \\ 
			{教育委員会} & \makecell[lt]{きょういく〜\\ いいんかい} & odbor za obrazovanje & imenica \\ 
			染める & そめる & obojati & glagol \\ 
			人権 & じんけん & ljudska prava & imenica, no-pridjev \\
		\bottomrule
	\end{xltabular}
	
	\clearpage
	
	\subsection*{Zadaci}
	\begin{enumerate}
 		\item Sažmite tekst u najviše dvije rečenice.
		\item Razgovarajte o tekstu.
	\end{enumerate}

	\subsection*{Domaća zadaća}
	\begin{enumerate}
		\item Napišite kratku priču ili par rečenica koristeći barem 5 riječi iz teksta. \\
		Rečenice ili tekst ne moraju nužno biti vezane uz samu vijest. 
		\item Odgovorite na sljedeća pitanja
		\begin{enumerate}
			\item 生徒は何をしなければならない?
			\item 証明書とともに生徒はどのような\f{証拠}{しょうこ}を\f{渡}{わた}せばならないのですか?
			\item \sloppy\p{本文の}\p{中に}\p{髪が}\p{茶色の}\p{生徒に}\p{ついて}\p{二つの}\p{意見が}\p{あります、}\p{その}\p{二つの}\p{意見を}\p{自分の}\p{\f{言葉}{ことば}で}\p{説明}\p{して}\p{ください。}
		\end{enumerate}
		\item Nadopunite sljedeće rečenice riječima iz vokabulara:
		\begin{enumerate}
			\item 花子ちゃんは\ansline{}ず自分の髪を\ansline{}めないと決めました。
			\item おばあさんは三つ\ansline{}の\f{携帯}{けいたい}のボタンを\ansline{}えた。
			\item 写真の\f{撮}{と}り方を\ansline{}しなければ\ansline{}になる\f{可能性}{かのうせい}があります。
			\item\sloppy\p{動物の}\p{\f{癖}{くせ}を}\p{\ansline{}べたいと}\p{思った}\p{鈴木さんは}\p{見つけ出した}\p{ことを}\p{ノートに}\p{\ansline{}く}\p{ことに}\p{しました。}
			\item\p{私は}\p{こう}\p{\ansline{}えた}\p{「他の}\p{人の}\p{\ansline{}}\p{なんて}\p{知らない、}\p{そんな}\p{こと}\p{\ansline{}ない」、}\p{だけど}\p{今は}\p{そう}\p{思わない、}\p{\ansline{}}\p{自体が}\p{ばか}\p{じゃないと}\p{知った}\p{から。}
		\end{enumerate}
	\end{enumerate}
\end{document}