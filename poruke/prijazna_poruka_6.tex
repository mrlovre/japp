\documentclass[intermediate]{grampig}
\usepackage{setspace}

\setlength{\topsep}{0em}
\setlength{\parskip}{1em}

\begin{document}
	\onehalfspacing
	{\Large \e{Gramatički kvizovi}} \\[-0.5em]
	
	Nakon što smo se prethodnih tjedana upoznali s par gramatičkih formi, vrijeme je da vidimo možemo li ih upotrijebiti i u praksi!\vspace{1em}
	
	U tome će nam poslužiti povremeni gramatički kvizovi, odnosno mozgalice koje treba riješiti prikladnom gramatikom:\vspace{1em}
	\begin{enumerate}\singlespacing
		\item \f{明日}{あした}は\f{友達}{ともだち}と\e{\f{買}{か}い\ansline{}\f{行}{い}きます}。 \hfill \e{ići kupovati} \vspace{0.5em}
		\item \f{子供}{こども}だった\f{時}{とき}は\e{\f{白雪姫}{しらゆきひめ}\ansline{}}\f{物語}{ものがたり}を\f{読}{よ}んだ。 \hfill \e{koja se zove}
	\end{enumerate}\vspace{1em}
	
	Umjesto izmišljanja primjera, treba nadopuniti rečenice nekom od gramatika iz prethodnih tjedana\textsuperscript{*}\footnotetext{\textsuperscript{*}u primjeru su neke druge zato jer ove čuvamo za sutrašnji kviz}.
	Natuknica s desne strane govori što tražena gramatika znači.
%	Ponekad će biti moguće i više od jednog odgovora --- prihvaćaju se svi koji štimaju po smislu i gramatici.
	\vspace{1em}
	
	\e{P. S.} Uz svaku rečenicu dobijete i hint (ispod slike) --- prijevod na hrvatski koji Vam može pomoći ako zapnete :)\vspace{1em}
	\begin{center}
		\e{Prvi gramatički kviz dolazi već sutra!}
	\end{center}
	
\end{document}