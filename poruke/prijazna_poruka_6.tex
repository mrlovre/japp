\documentclass[intermediate]{grampig}
\usepackage{setspace}

\setlength{\topsep}{0em}
\setlength{\parskip}{1em}

\begin{document}
	\onehalfspacing
	{\Large \e{Gramatički kvizovi}} \\[-0.5em]
	
	Nakon što smo se prethodnih tjedana upoznali s par gramatičkih formi, vrijeme je da vidimo možemo li ih upotrijebiti i u praksi!\vspace{1em}
	
	U tome će nam poslužiti povremeni gramatički kvizovi, odnosno mozgalice koje treba riješiti prikladnom gramatikom. Npr.\vspace{1em}
	\begin{center}
		\begin{tabular}{r@{\hspace{2em}}l}
		\f{明日}{あした}は\f{友達}{ともだち}と\e{\f{買}{か}い\ansline{}}\f{行}{い}きます。 & \e{ići kupovati} \br
		\f{子供}{こども}だった\f{時}{とき}は\e{\f{白雪姫}{しらゆきひめ}\ansline{}}\f{物語}{ものがたり}を\f{読}{よ}んだ。 & \e{koja se zove} \\
		\end{tabular}
	\end{center}\vspace{1em}
	
	Ovaj put umjesto smišljanja primjera, trebate samo nadopuniti rečenice nekom od gramatika iz prethodnih tjedana\textsuperscript{*}\footnotetext{\textsuperscript{*}u primjeru su neke druge zato jer ove čuvamo za pravi kviz}.
%	Ovaj put umjesto da sami pišete primjere upotrebljavajući zadane gramatičke forme, imate skoro gotovu rečenicu koju trebate nadopuniti odgovarajućom gramatikom.
	Natuknica s desne strane sugerira što traženi oblik mora implicirati.
	\vspace{1em}
	
	\e{P. S.} Uz rečenicu dobijete i hint ispod slike --- prijevod na hrvatski, koji Vas može izvući ako zapnete :)\vspace{1em}
	\begin{center}
		\e{Prvi gramatički kviz dolazi već sutra!}
	\end{center}
	
\end{document}