\documentclass[ocha-dark]{grampig}
\usepackage{setspace}

\setlength{\topsep}{0em}
\setlength{\parsep}{0em}

\begin{document}
	\onehalfspacing
	\obeylines
	{\Large \e{漢字・かんじ i kako ih preživjeti}} \\[-0.5em]
	
	Ove srijede očajavamo nad kanjijima/kamđijima/kanđijima, kako god da se zvali. \\[-1em]
	
	\begin{center}
		Sjećamo se kako je sve počelo:
		Šta je malo kamđija nakon hiragane i katakane, rekli su...
		Vidit ćeš, bit će zabavno, rekli su...
	\end{center}\vspace{0.5em}
	
	\begin{center}
		\e{Jesu li nas lagali?}
		Da li je jednostavno? Teško da jest. 
		Da li je zabavno? Zna biti, zna biti. 
		Da li vas ponekad baca u očaj? Saznajmo zajedno. 
	\end{center}\vspace{0.5em}
	
	Kao i uvijek, kava/čaj i keksi dolaze s vama, dok očaj vjerno vreba. Očaj, a možda i 鬱 \\[-1em]
	
	\begin{center}\e{
			U srijedu, 14.4. u 19.00 sati na Discordu. \\
			Svi su pozvani i dobrodošli!
		}
	\end{center}
\end{document}