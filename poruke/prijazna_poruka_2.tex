\documentclass[pig]{grampig}
\usepackage{ragged2e}
\usepackage{setspace}

\newcolumntype{L}{l}
\newcolumntype{R}{>{\color{\accentcolor}}r}
\newcolumntype{C}{>{\itshape}c}

\begin{document}
	\begin{minipage}{\width}
		\onehalfspacing
		{\Large \e{PIG Turnir}}  \\[-0.5em]
		
		Novost na Makoto tržištu! 
		Ne dva, ne tri, već četiri u jedan, a možda čak i više! Sve ovisi o Vama!  \\[-0.5em]
		
		Ogledajte se u sklapanju rečenice, jedne ili više njih, možda čak i sastava!
		Jedino pravilo jest da upotrijebite sve tri riječi, \e{pridjev-imenicu-glagol}, ne nužno u gramatičkom formatu u kojem su napisane: negirajte ih, bacajte u prošlost, radite im što god Vas volja. 
		Riječ na katakani dobivate gratis, možete je, ali i ne morate upotrijebiti.  \\[-0.5em]
		
		Vi samo pišite --- naše je da čitamo.
		Vi samo pitajte --- naše je da odgovaramo. 
		Bojite se grešaka? Ne bojte se!
		Tko radi, taj i griješi. 
		A tko griješi, taj i uči. \\[-1.5em]
		
%		\begin{table}
%			\centering
%			\textsc{\e{Primjer:}} \br
%			\begin{tabular}{RCl}
%				青い & i-pridjev & plavo \\
%				\ruby{空}{そら} & imenica & nebo \\
%				\ruby{焼}{や}く & u-glagol & peći \\
%				\f{ハンバーガ}{\strut} & katakana & hamburger \\
%			\end{tabular}
%		\end{table} \vspace{-1em}
		
%		\singlespacing
%		\begin{itemize}
%			\item 青い空の下でハンバーガーを焼かない。\bh
%			Ispod plavog neba ne pečem hamburger. \\[-0.5em]
%			
%			\item 青くない空を見て、パンを焼く。\bh
%			Gledam neplavo nebo i pečem kruh. \\[-0.5em]
%			
%			\item 空の下で、青い\f{電車}{でんしゃ}に乗っていた。({電車} = vlak)\\
%			家に帰って、ソーセージを焼いた。\bh
%			Ispod neba, vozio sam se u plavom vlaku . \\
%			Kad sam došao kući, pekao sam kobasice. 
%		\end{itemize}
	\end{minipage}
\end{document}