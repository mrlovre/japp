%!TeX program = xelatex
\documentclass[a4paper, 12pt]{amsart}
\usepackage[croatian]{babel}
\usepackage[PunctStyle=banjiao,CJKchecksingle,CJKnumber,CJKspace=true]{xeCJK}

\setCJKmainfont[Scale=MatchUppercase, BoldFont={* SemiBold}]{Source Han Serif JP}
\setCJKsansfont[Scale=MatchUppercase, BoldFont={* Bold}]{Source Han Sans JP}
\setmainfont[BoldFont={* Semibold}]{Source Serif Pro}

\usepackage{indentfirst}
\usepackage{booktabs}
\usepackage{setspace}
\usepackage{enumitem}
\usepackage[defaultlines=4, all]{nowidow}

\usepackage[margin=3cm]{geometry}

\usepackage[overlap,CJK]{ruby}
\renewcommand{\rubysep}{-0.8ex}
\renewcommand{\rubysize}{0.42}
\newcommand{\f}[2]{\ruby{#1}{\smash{\sffamily{#2}}}\CJKglue}
%\renewcommand{\f}{}

\usepackage[babel=true]{microtype}
%\setlist[itemize]{wide, itemsep=0.0em, topsep=0.0em, parsep=0.0em, labelwidth=!, labelindent=0.5ex, leftmargin=1ex}
%\setlist[enumerate]{wide, itemsep=0.0em, topsep=0.0em, parsep=0.0em, labelwidth=!, labelindent=0.5ex, leftmargin=1ex}

\title{Sažetak japanske gramatike za ponavljanje}
\author{Lovriša}

\onehalfspacing

\begin{document}
	\maketitle
	
	\section{Uvod}
	qwertyuiopsdfghjklzxcvbnmQWERTYUIOPASDFGHJKLZXCVBNM \\
	dfjkl \\
	thanks foj ingtroducing vphantom that I haven't known before. It will be sophisticatedMl\f{未}{み}\f{然}{ぜん}\f{形}{けい} jif on the fly we can make a query against the letters to find the heighest letter and the deepest letter. 
	
	\section{Vrste riječi}
	\begin{itemize}
		\item 名詞「めいし」 --- imenice
		\item 形容詞「けいようし」 --- pridjevi
		\item 動詞「どうし」 --- glagoli
		\item 副詞「ふくし」 --- prilozi
		\item 助詞「じょし」 --- čestice
	\end{itemize}
  \subsection{\f{名}{めい}\f{詞}{し} --- imenice}

  \section{Osnovni djelovi japanske rečenice}
  \begin{itemize}
  	\item 
  \end{itemize}

  \section{Konjugirane osnove}
  Osnova je korijen.
  Oblik je ono što se koristi u rečenicama.
 
  
  \begin{center}
  	\begin{tabular}{ccccccccccc}
  		あ & か & さ & た & な & は & ま & や & ら & わ & \\
  		い & き & し & ち & に & ひ & み &   & り & ゐ & \\
  		う & く & す & つ & ぬ & ふ & む & ゆ & る & わ & ん \\
  		え & け & せ & て & ね & へ & め &   & れ & ゑ & \\ 
  		お & こ & そ & と & の & ほ & も & よ & ろ & を &  \\
  	\end{tabular}
  \end{center}

  \subsection{\f{未}{み}\f{然}{ぜん}\f{形}{けい}\ --- \textit{irrealis} osnova}
  Ova osnova dobiva se kod 五段 glagola pomicanjem zadnje \textit{kane} u \textit{a}-redak.
  U literaturi se još naziva i \textit{irrealis}, jer se tradicionalno koristila u oblicima koji su značili da se radnja neće nije dogodila, ili da se nije još dogodila.
  U modernom jeziku i dalje se koristi za negativne oblike, ali također i za pasivni i kauzativni oblik.
  
  Iz nje se izvode:
  \begin{itemize}
  	\item ない形, ず形 --- negativni \textit{nai} i \textit{zu} oblici,
    \item 受身形 --- pasivni oblik
  \end{itemize}

  \subsection{\f{意}{い}\f{志}{し}\f{形}{けい}\ --- volicionalna osnova}
  Ova osnova dobiva se kod 五段 glagola pomicanjem zadnje \textit{kane} u \textit{o}-redak.
  Izvedena je iz \f{未}{み}\f{然}{ぜん}\f{形}{けい} ili \textit{irrealis} osnove modernom jeziku kao rezultat glasovne promjene kod koje \textit{a} pored \textit{u} u kombinaciji \textit{au} prelazi u \textit{o} i time nastaje produženo \textit{\=o}.
  Nalazi se samo u volicionalu (hortativu).
  
  \subsection{\f{連}{れん}\f{用}{よう}\f{形}{けい}\ --- vezivna osnova}
  Ova osnova dobiva se kod 五段 glagola pomicanjem zadnje \textit{kane} u \textit{i}-redak.
  
  \subsection{\f{音}{おん}\f{便}{びん}\f{形}{けい}\ --- eufonijska osnova}
  
  \subsection{\f{終}{しゅう}\f{止}{し}\f{形}{けい}\ --- terminativna osnova}
  
  
  \subsection{\f{連}{れん}\f{体}{たい}\f{形}{けい}\ --- atributivna osnova}
  
  \subsection{\f{仮}{か}\f{定}{てい}\f{形}{けい}\ --- hipotetska osnova}
  
  \subsection{\f{可}{か}\f{能}{のう}\f{形}{けい}\ --- potencijalna osnova}
    
  \subsection{\f{命}{めい}\f{令}{れい}\f{形}{けい}\ --- zapovijedna osnova}
  

  \section{Glagoli}
  \section{Čestice}
  
\end{document}