%!TeX program = xelatex
\documentclass[a4paper, 12pt]{amsart}
\usepackage[croatian]{babel}
%\usepackage[PunctStyle=CCT,CJKchecksingle,CJKnumber,CJKspace=true]{xeCJK}
\usepackage{zxjatype}

\setCJKmainfont[Scale=MatchUppercase, BoldFont={* SemiBold}, ItalicFont=*]{Source Han Serif JP}
\setCJKsansfont[Scale=MatchUppercase, BoldFont={* Bold}, ItalicFont=*]{Source Han Sans JP}
\setmainfont[BoldFont={* Semibold}]{Source Serif Pro}

\usepackage{booktabs}
\usepackage{setspace}
\usepackage{enumitem}
\usepackage[defaultlines=4, all]{nowidow}

\usepackage[margin=3cm]{geometry}
\usepackage{pbox}

\usepackage[overlap,CJK]{ruby}
\renewcommand{\rubysep}{-3.7ex}
\renewcommand{\rubysize}{0.54}
\newcommand{\f}[2]{\ruby{#1}{\sffamily\mdseries\protect\furiganafix{#2}}\CJKglue}

\usepackage{xstring}
\newcommand{\squeeze}{\kern -0.2em}
\newcommand{\furiganafix}[1]{{%
    \StrSubstitute{#1}{・}{@・@}[\x]%
    \StrSubstitute{\x}{ゃ}{@ゃ@}[\x]%
    \StrSubstitute{\x}{ゅ}{@ゅ@}[\x]%
    \StrSubstitute{\x}{ょ}{@ょ@}[\x]%
    \StrSubstitute{\x}{ぁ}{@ぁ@}[\x]%
    \StrSubstitute{\x}{ぃ}{@ぃ@}[\x]%
    \StrSubstitute{\x}{ぅ}{@ぅ@}[\x]%
    \StrSubstitute{\x}{ぇ}{@ぇ@}[\x]%
    \StrSubstitute{\x}{ぉ}{@ぉ@}[\x]%
    \StrSubstitute{\x}{っ}{@っ@}[\x]%
    \StrSubstitute{\x}{ャ}{@ャ@}[\x]%
    \StrSubstitute{\x}{ュ}{@ュ@}[\x]%
    \StrSubstitute{\x}{ョ}{@ョ@}[\x]%
    \StrSubstitute{\x}{ァ}{@ァ@}[\x]%
    \StrSubstitute{\x}{ィ}{@ィ@}[\x]%
    \StrSubstitute{\x}{ゥ}{@ゥ@}[\x]%
    \StrSubstitute{\x}{ェ}{@ェ@}[\x]%
    \StrSubstitute{\x}{ォ}{@ォ@}[\x]%
    \StrSubstitute{\x}{ッ}{@ッ@}[\x]%
    \StrSubstitute{\x}{@@}{@}[\x]%
    \StrSubstitute{\x}{@}{\squeeze}[\x]%
    \x}}

\newcommand{\g}{\furiganafix}

\usepackage[babel=true]{microtype}
%\setlist[itemize]{wide, itemsep=0.0em, topsep=0.0em, parsep=0.0em, labelwidth=!, labelindent=0.5ex, leftmargin=1ex}
%\setlist[enumerate]{wide, itemsep=0.0em, topsep=0.0em, parsep=0.0em, labelwidth=!, labelindent=0.5ex, leftmargin=1ex}
\usepackage{makecell}

\title{Sažetak japanske gramatike za ponavljanje}
\author{Lovriša}

\onehalfspacing

\begin{document}
  \maketitle

  \section{Uvod}
  qwertyuiopsdfghjklzxcvbnmQWERTYUIOPASDFGHJKLZXCVBNM \\
  dfjkl \\
  thanks foj ingtroducing vphantom that I haven't known before. It will be sophisticatedMl\f{未然}{みぜん}形jif sophisticatedMl\f{未然形}{みぜんけい}jif sophisticatedMl\f{未}{み}\f{然}{ぜん}\f{形}{けい}jifon the fly we can make a query against the \g{しゅう・ぎゅっと・しょう} letters to find the heighest letter and the deepest letter.

  \section{Vrste riječi}
  Riječi se dijele na sljedeće vrste:
  \begin{itemize}
  	\item nepromjenjive vrste:
	  \begin{itemize}
	    \item \f{名詞}{めい・し} --- imenice
	    \item \f{形容詞}{けい・じょう・し} --- \textit{na}-pridjevi
	    \item \f{副詞}{ふく・し} --- prilozi
	  \end{itemize}
	  \item promjenjive vrste:
	   \begin{itemize}
	  	\item \f{動詞}{どう・し} --- glagoli
	  	\item \f{形容詞}{けい・よう・し} --- \textit{i}-pridjevi
	  	\item kopula
	  \end{itemize}
	  \item ostalo:
	   \begin{itemize}
	  	\item \f{助詞}{じょ・し} --- čestice
	  \end{itemize}
  \end{itemize}

  Nema prijedloga kao u hrvatskom (u, na, pod).
  Umjesto toga, koriste se imenice:
  \begin{itemize}
  	\item 机の上にパソコンがある。 \textit{Računalo je na stolu.}
  \end{itemize}

  \subsection{\f{名詞}{めい・し} --- imenice}


  \subsection{\f{助詞}{じょ・し} --- čestice}

  \section{Struktura japanske rečenice}
  Jednostavna japanska rečenica sastoji se samo od subjekta i predikata.
  Primjeri:
  \begin{itemize}
  	\item (私が)\f{来}{く}る! \textit{(Ja) dolazim!}
  	\item (これが)\f{高}{たか}い! \textit{(Ovo) je skupo!}
  	\item (それが)\f{危}{あぶ}ないだ! \textit{(To) je opasno!}
  \end{itemize}
  
  Japanska rečenica sastoji se od \textit{teme} i \textit{reme} (engl. \textit{topic} i \textit{comment}).
  
  \subsection{Predikat}
  \textit{TODO: Što je predikat --- definicija.}
  Predikat može biti glagolski i imenski.
  Glagolski predikati sastoje se od glagola ili \textit{i}-pridjeva, a imenski od imenice ili \textit{na}-pridjeva i kopule.
  Predikat nosi osnovnu informaciju u rečenici, a ostale riječi pobliže objašnjavaju smisao rečenice.
  
  \subsection{Opisnice}
  Dijelovi rečenice koji pobliže opisuju imensku frazu.
  \begin{itemize}
  	\item \textit{i}-pridjev
  	\item \textit{na}-pridjev
  	\item \textit{no}-pridjev
  	\item imenica + の
  	\item \textit{adjectival}
  	\item zavisna rečenica
  \end{itemize}
  
  \subsection{Priložnice}
  Dijelovi rečenice koji pobliže opisuju predikat.
  \begin{itemize}
  	\item pravi prilozi
  	\item imenski prilozi
    \item \textit{i}-pridjev
    \item \textit{na}-pridjev
    \item \textit{no}-pridjev
    \item padežne čestice
  \end{itemize}

  \textit{TODO: duck-typeati ove vrste riječi i usporediti njihova svojstva.}
  Imenica, \textit{no}-pridjev, \textit{na}-pridjev, \textit{i}-pridjev, \textit{adjectival}.
  \begin{itemize}
  	\item opisuju imenicu --- svi.
  	\item može se koristiti kao predikat?
  	\item može se koristiti kao prilog?
  	\item može se koristiti kao samostalna imenica?
  \end{itemize}

  \subsection{Veze između vrsta riječi}
  Glagoli --- idjevi.
  Idjevi/nadjevi/nodjevi --- imenice.
  Idjevi/nadjevi/nodjevi --- prilozi.
  Imenice --- glagoli.
  Imenice --- prilozi.
  Zavisne rečenice --- imenice.
  
  \section{Glagoli}
  Osnova je korijen.
  Oblik je ono što se koristi u rečenicama.

  \subsection{Glagolske kategorije}

  \subsubsection{Aktivni i stativni}
  Aktivni glagoli \f{動態動詞}{どう・たい・どう・し} su oni kojima se izražava radnja ili zbivanje.
  Stativnim glagolima \f{状態動詞}{じょう・たい・どう・し} izražava se stanje.

  \subsubsection{Prijelazni i neprijelazni}
  \subsection{Konjugacijske skupine}
  \subsubsection{\f{五段活用}{ご・だん・かつ・よう} --- petostupanjska konjugacija}
  Glagoli koji pripadaju ovoj konjugaciji u rječničkom obliku završavaju nastavcima く, ぐ, す, ぬ, む, る ili う.
  Završetak ovih glagola se prilikom konjugiranja pomiče kroz svih pet redaka slogovne tablice, pa otud dolazi naziv \f{五段活用}{ご・だん・かつ・よう}.
  Povijesno se ova konjugacija zvala \f{四段活用}{よつ・だん・かつ・よう}, jer se \textasciitilde\textit{o} redak nije upotrebljavao.

  \subsubsection{\f{一段活用}{いち・だん・かつ・よう} --- jednostupanjska konjugacija}
  Glagoli koji pripadaju ovoj konjugaciji u rječničkom obliku završavaju nastavcima \textasciitilde\textit{i}る ili \textasciitilde\textit{e}る.
  Njihove konjugirane osnove gube završetak る, ali prethodna \textasciitilde\textit{i}, odnosno \textasciitilde\textit{e} \textit{kana} ostaje nepromijenjena, pa otud dolazi naziv \f{一段活用}{いち・だん・かつ・よう}.
  Dijeli se na gornju (\f{上一段}{かみ・いち・だん}) i donju (\f{下一段}{しも・いち・だん}), prema pozicijama \textasciitilde\textit{i} i \textasciitilde\textit{e} redaka slogovne tablice.

  \subsubsection{\f{変格活用}{へん・かく・かつ・よう} --- nepravilna konjugacija}

  \subsubsection{\f{二段活用}{に・だん・かつ・よう} --- dvostupanjska konjugacija}
  U modernom japanskom ova konjugacija se koristi samo još kod glagola \f{得}{え}る.
  U klasičnom japanskom bilo je puno više takvih glagola, koji su danas postali dio jednostupanjske konjugacijske skupine.
  Glagoli koji su pripadali ovoj konjugaciji u rječničkom obliku završavali su bilo kojim nastavkom \textasciitilde\textit{u}, osim ぷ.
  Nadalje, ovisno o tome jesu li pripadali \f{上}{かみ} ili \f{下}{しも} podskupini, njihove \textit{irrealis} (\f{未然形}{み・ぜん・けい}) i vezivne (\f{連用形}{れん・たい・けい}) osnove dobivale bi se pomicanjem završne \textit{kane} u odgovarajući \textasciitilde\textit{i}, odnosno \textasciitilde\textit{e} redak; iznimno, ゆ se mijenjalo u い/え, a う u ゐ/ゑ --- dvije \textit{kane} koje se u modernom japanskom više ne upotrebljavaju.
  Također, ovo je i jedina konjugacijska skupina glagola kod koje su se opisni i završni oblici razlikovali: završni oblik bio je jednak rječničkom, a opisni je dobivao nastavak る.
  \textit{Realis} i imperativna osnova dobivale su se na isti način kao i kod glagola jednostupanjske konjugacije.

  \subsection{Konjugirane osnove}
  \begin{center}
    \begin{tabular}{lllllllllll}
    	& \multicolumn{1}{c}{\textit{\o\textasciitilde}} & \multicolumn{1}{c}{\textit{k\textasciitilde}} & \multicolumn{1}{c}{\textit{s\textasciitilde}} & \multicolumn{1}{c}{\textit{t\textasciitilde}} & \multicolumn{1}{c}{\textit{n\textasciitilde}} & \multicolumn{1}{c}{\textit{h\textasciitilde}} & \multicolumn{1}{c}{\textit{m\textasciitilde}} & \multicolumn{1}{c}{\textit{y\textasciitilde}} & \multicolumn{1}{c}{\textit{r\textasciitilde}} & \multicolumn{1}{c}{\textit{w\textasciitilde}} \\
      \multicolumn{1}{c}{\textit{\textasciitilde a}} & あ & か & さ & た & な & は & ま & や & ら & わ \\
      \multicolumn{1}{c}{\textit{\textasciitilde i}} & い & き & し & ち & に & ひ & み & ・ & り & ゐ \\
      \multicolumn{1}{c}{\textit{\textasciitilde u}} & う & く & す & つ & ぬ & ふ & む & ゆ & る & ・ \\
      \multicolumn{1}{c}{\textit{\textasciitilde e}} & え & け & せ & て & ね & へ & め & ・ & れ & ゑ \\
      \multicolumn{1}{c}{\textit{\textasciitilde o}} & お & こ & そ & と & の & ほ & も & よ & ろ & を \\
    \end{tabular}
  \end{center}

  \subsubsection{\f{未然形}{み・ぜん・けい} --- \textit{irrealis} osnova}
  Ova osnova dobiva se kod 五段 glagola pomicanjem zadnje \textit{kane} rječničkog oblika iz \textit{u}-retka u \textit{a}-redak, a kod \f{一段}{いち・だん} glagola odbacivanjem završetka る.
  U literaturi se još naziva i \textit{irrealis}, jer se tradicionalno koristila u oblicima koji su značili da se radnja nije (još) dogodila.
  U modernom jeziku i dalje se koristi za negativne oblike, ali također i za pasivni i kauzativni oblik.

  Iz nje se izvode:
  \begin{itemize}
    \item ない形, ず形 --- negativni \textit{nai} i \textit{zu} oblici,
    \item 受身形 --- pasivni oblik,
  \end{itemize}
  a u klasičnom japanskom i:
  \begin{itemize}
    \item 〜ば --- kondicional.
  \end{itemize}


  \subsubsection{\f{意志形}{い・し・けい} --- volicionalna osnova}
  Ova osnova dobiva se kod 五段 glagola pomicanjem zadnje \textit{kane} rječničkog oblika  iz \textit{u}-retka u \textit{o}-redak, a kod \f{一段}{いち・だん} glagola odbacivanjem završetka る.
  Izvedena je iz \f{未}{み}\f{然}{ぜん}\f{形}{けい} ili \textit{irrealis} osnove u modernom jeziku kao rezultat glasovne promjene kod koje \textit{a} pored \textit{u} u kombinaciji \textit{au} prelazi u \textit{o} i time nastaje produženo \textit{\=o}.
  Nalazi se samo u volicionalu (hortativu).

  \subsubsection{\f{連用形}{れん・よう・けい} --- vezivna osnova}
  Ova osnova dobiva se kod 五段 glagola pomicanjem zadnje \textit{kane} rječničkog oblika u \textit{i}-redak.

  Iz nje se izvode:
  \begin{itemize}
    \item pristojni oblici 〜ます
    \item složeni glagoli
  \end{itemize}

  \subsubsection{\f{音便形}{おん・びん・けい} --- eufonijska osnova}
  Ova osnova izvedena je \f{連}{れん}\f{用}{よう}\f{形}{けい} ili vezivne osnove u modernom jeziku kao rezultat glasovne promjene kod koje kombinacija \textit{\textasciitilde\!~i} + \textit{t\!~\textasciitilde} kontrakcijom postaje
  \begin{center}
    \begin{tabular}{rcl}
      い/ち/り + \textit{t}\textasciitilde & \textrightarrow & っ\textit{t}\textasciitilde \\
      び/み/に + \textit{t}\textasciitilde & \textrightarrow & ん\textit{d}\textasciitilde \\
      き + \textit{t}\textasciitilde & \textrightarrow & い\textit{t}\textasciitilde \\
      ぎ + \textit{t}\textasciitilde & \textrightarrow & い\textit{d}\textasciitilde \\
    \end{tabular}
  \end{center}

  \subsubsection{\f{終止形}{しゅう・し・けい} --- terminativna osnova}
  Ova osnova jednaka je rječničkom obliku i završava na \textit{u}.
  Koristi se za završetke rečenica.

  \subsubsection{\f{連体形}{れん・たい・けい} --- atributivna osnova}
  Ova osnova jednaka je rječničkom obliku i završava na \textit{u}.
  Koristi se za opis imenskih fraza.

  \subsubsection{\f{仮定形}{か・てい・けい} --- hipotetska osnova}
  Ova osnova dobiva se kod 五段 glagola pomicanjem zadnje kane rječničkog oblika iz \textit{u}-retka u \textit{e}-redak.
  Naziva se tradicionalno još i \textit{realis} osnova, jer se u klasičnom japanskom koristila u oblicima kod kojih se radnja već dogodila.

  \subsubsection{\f{可能形}{か・のう・けい} --- potencijalna osnova}

  \subsubsection{\f{命令形}{めい・れい・けい} --- zapovijedna osnova}

  \begin{center}\renewcommand{\rubysep}{-3.25ex}
  	\begin{tabular}{rllllllll}
  		\toprule
 			& \f{語例}{ごれい} & \f{未然形}{み・ぜん・けい} & \f{意志形}{い・し・けい} & \f{連用形}{れん・よう・けい} & \f{音便形}{おん・びん・けい} & \makecell[lc]{\f{終止形}{しゅう・し・けい} \\ \f{連体形}{れん・たい・けい}} & \makecell[lc]{\f{仮定形}{か・てい・けい} \\ \f{可能形}{か・のう・けい}} & \f{命令形}{めい・れい・けい}\\ \midrule
  		  上一段 & \f{見}{み}る & \f{見}{み}  & \f{見}{み}  & \f{見}{み}  & \f{見}{み}    & \f{見}{み}る & \f{見}{み}れ & \makecell[lt]{\f{見}{み}ろ\\\f{見}{み}よ} \\
  		  下一段 & \f{寝}{ね}る & \f{寝}{ね}  & \f{寝}{ね}  & \f{寝}{ね}  & \f{寝}{ね}    & \f{寝}{ね}る & \f{寝}{ね}れ & \makecell[lt]{\f{寝}{ね}ろ\\\f{寝}{ね}よ} \\ \midrule
  		カ行五段 & \f{書}{か}く & \f{書}{か}か & \f{書}{か}こ & \f{書}{か}き & \f{書}{か}い  & \f{書}{か}く & \f{書}{か}け & \f{書}{か}け \\
  		ガ行五段 & \f{嗅}{か}ぐ & \f{嗅}{か}が & \f{嗅}{か}ご & \f{嗅}{か}ぎ & \f{嗅}{か}い\textsuperscript{*} & \f{嗅}{か}ぐ & \f{嗅}{か}げ & \f{嗅}{か}げ \\
  		サ行五段 & \f{消}{け}す & \f{消}{け}さ & \f{消}{け}そ & \f{消}{け}し & \f{消}{け}し & \f{消}{け}す & \f{消}{け}せ & \f{消}{け}せ \\
  		タ行五段 & \f{持}{も}つ & \f{持}{も}た & \f{持}{も}と & \f{持}{も}ち & \f{持}{も}っ & \f{持}{も}つ & \f{持}{も}て & \f{持}{も}て \\
  		バ行五段 & \f{飛}{と}ぶ & \f{飛}{と}ば & \f{飛}{と}ぼ & \f{飛}{と}び & \f{飛}{と}ん\textsuperscript{*} & \f{飛}{と}ぶ & \f{飛}{と}べ & \f{飛}{と}ね \\
  		マ行五段 & \f{飲}{の}む & \f{飲}{の}ま & \f{飲}{の}も & \f{飲}{の}み & \f{飲}{の}ん\textsuperscript{*} & \f{飲}{の}む & \f{飲}{の}め & \f{飲}{の}め \\
  		ナ行五段 & \f{死}{し}ぬ & \f{死}{し}ぬ & \f{死}{し}の & \f{死}{し}に & \f{死}{し}ん\textsuperscript{*} & \f{死}{し}ぬ & \f{死}{し}ね & \f{死}{し}ね \\
  		ラ行五段 & \f{取}{と}る & \f{取}{と}ら & \f{取}{と}ろ & \f{取}{と}り & \f{取}{と}っ & \f{取}{と}る & \f{取}{と}れ & \f{取}{と}れ \\
		  ワア行五段 & \f{買}{か}う & \f{買}{か}わ & \f{買}{か}お & \f{買}{か}い & \f{買}{か}っ & \f{買}{か}う & \f{買}{か}え & \f{買}{か}え \\ \midrule
  		 カ行変 & \f{来}{く}る & \f{来}{こ} & \f{来}{こ} & \f{来}{き} & \f{来}{き} & \f{来}{く}る & \makecell[lt]{\f{来}{く}れ\\\f{来}{こ}} & \f{来}{こ}い \\
  		 サ行変 & する         & \makecell[lt]{し\\さ\\せ} & し & し & し & する & \makecell[lt]{すれ\\できる} & \makecell[lt]{しろ\\せよ} \\
  		\bottomrule
  	\end{tabular}
  \end{center}

  \subsection{Glagolski oblici}

  \subsubsection{Imperfektiv}
  Označava generalno nesvršenu radnju ili stanje.
  Jednak je terminativnoj i atributivnoj osnovi.

  \subsubsection{Negativ}
  Označava stanje u kojem se neka radnja ili stanje ne odvija.

  \subsubsection{Perfektiv}
  Označava svršenu radnju ili stanje.

  \subsubsection{\textit{\textasciitilde{}te} oblik}
  Koristi se za spajanje.

  \subsubsection{Konjunktiv}
  Koristi se za spajanje.

  \subsubsection{Volicional}
  Označava poziv da se učini radnja.

  \subsubsection{Pasiv}
  Označava da subjekt trpi posljedicu agentove radnje.

  \subsubsection{Kauzativ}
  Označava da je subjekt uzrokuje objekt da čini radnju.

  \subsubsection{Imperativ}
  Označava jaku zapovijed.

  \subsubsection{Potencijal}
  Označava moguće stanje.

  \subsubsection{Kondicional}
  Označava uvjet u zavisnoj rečenicu koji ostvaruje posljedicu iz glavne rečenice kada je ispunjen.

  \section{Čestice}

\end{document}
