%!TeX program = xelatex
\documentclass[a4paper, 12pt]{amsart}
\usepackage[croatian]{babel}
%\usepackage[PunctStyle=CCT,CJKchecksingle,CJKnumber,CJKspace=true]{xeCJK}
\usepackage{zxjatype}

\setCJKmainfont[Scale=MatchUppercase, BoldFont={* SemiBold}, ItalicFont=*]{Source Han Serif JP}
\setCJKsansfont[Scale=MatchUppercase, BoldFont={* Bold}, ItalicFont=*]{Source Han Sans JP}
\setmainfont[BoldFont={* Semibold}]{Source Serif Pro}

\usepackage{booktabs}
\usepackage{setspace}
\usepackage{enumitem}
\usepackage[defaultlines=4, all]{nowidow}

\usepackage[margin=3cm]{geometry}

\usepackage[overlap,CJK]{ruby}
\renewcommand{\rubysep}{-3.8ex}
\renewcommand{\rubysize}{0.56}
\newcommand{\f}[2]{\ruby{#1}{\sffamily\protect\furiganafix{#2}}\CJKglue}

\usepackage{xstring}
\newcommand{\squeeze}{\kern -0.2em}
\newcommand{\furiganafix}[1]{{%
    \StrSubstitute{#1}{・}{@・@}[\x]%
    \StrSubstitute{\x}{ゃ}{@ゃ@}[\x]%
    \StrSubstitute{\x}{ゅ}{@ゅ@}[\x]%
    \StrSubstitute{\x}{ょ}{@ょ@}[\x]%
    \StrSubstitute{\x}{ぁ}{@ぁ@}[\x]%
    \StrSubstitute{\x}{ぃ}{@ぃ@}[\x]%
    \StrSubstitute{\x}{ぅ}{@ぅ@}[\x]%
    \StrSubstitute{\x}{ぇ}{@ぇ@}[\x]%
    \StrSubstitute{\x}{ぉ}{@ぉ@}[\x]%
    \StrSubstitute{\x}{っ}{@っ@}[\x]%
    \StrSubstitute{\x}{ャ}{@ャ@}[\x]%
    \StrSubstitute{\x}{ュ}{@ュ@}[\x]%
    \StrSubstitute{\x}{ョ}{@ョ@}[\x]%
    \StrSubstitute{\x}{ァ}{@ァ@}[\x]%
    \StrSubstitute{\x}{ィ}{@ィ@}[\x]%
    \StrSubstitute{\x}{ゥ}{@ゥ@}[\x]%
    \StrSubstitute{\x}{ェ}{@ェ@}[\x]%
    \StrSubstitute{\x}{ォ}{@ォ@}[\x]%
    \StrSubstitute{\x}{ッ}{@ッ@}[\x]%
    \StrSubstitute{\x}{@@}{@}[\x]%
    \StrSubstitute{\x}{@}{\squeeze}[\x]%
    \x}}

\newcommand{\g}{\furiganafix}

\usepackage[babel=true]{microtype}
%\setlist[itemize]{wide, itemsep=0.0em, topsep=0.0em, parsep=0.0em, labelwidth=!, labelindent=0.5ex, leftmargin=1ex}
%\setlist[enumerate]{wide, itemsep=0.0em, topsep=0.0em, parsep=0.0em, labelwidth=!, labelindent=0.5ex, leftmargin=1ex}

\title{Sažetak japanske gramatike za ponavljanje}
\author{Lovriša}

\onehalfspacing

\begin{document}
  \maketitle

  \section{Uvod}
  qwertyuiopsdfghjklzxcvbnmQWERTYUIOPASDFGHJKLZXCVBNM \\
  dfjkl \\
  thanks foj ingtroducing vphantom that I haven't known before. It will be sophisticatedMl\f{未然}{みぜん}形jif sophisticatedMl\f{未然形}{みぜんけい}jif sophisticatedMl\f{未}{み}\f{然}{ぜん}\f{形}{けい}jifon the fly we can make a query against the \g{しゅう・ぎゅっと・しょう} letters to find the heighest letter and the deepest letter.

  \section{Vrste riječi}
  \begin{itemize}
    \item 名詞「めいし」 --- imenice
    \item 形容詞「けいようし」 --- pridjevi
    \item 動詞「どうし」 --- glagoli
    \item 副詞「ふくし」 --- prilozi
    \item 助詞「\g{じょし}」 --- čestice
  \end{itemize}
  \subsection{\f{名詞}{めい・し} --- imenice}

  \section{Osnovni djelovi japanske rečenice}
  \begin{itemize}
    \item
  \end{itemize}

  \section{Glagoli}
  Osnova je korijen.
  Oblik je ono što se koristi u rečenicama.

  \subsection{Glagolske kategorije}

  \subsubsection*{Aktivni i stativni}

  \subsubsection*{Prijelazni i neprijelazni}

  \subsection{Konjugirane osnove}
  \begin{center}
    \begin{tabular}{ccccccccccc}
      あ & か & さ & た & な & は & ま & や & ら & わ & \\
      い & き & し & ち & に & ひ & み &   & り & ゐ & \\
      う & く & す & つ & ぬ & ふ & む & ゆ & る & わ & ん \\
      え & け & せ & て & ね & へ & め &   & れ & ゑ & \\
      お & こ & そ & と & の & ほ & も & よ & ろ & を &  \\
    \end{tabular}
  \end{center}

  \subsubsection*{\f{未然形}{み・ぜん・けい} --- \textit{irrealis} osnova}
  Ova osnova dobiva se kod 五段 glagola pomicanjem zadnje \textit{kane} iz \textit{u}-retka u \textit{a}-redak.
  U literaturi se još naziva i \textit{irrealis}, jer se tradicionalno koristila u oblicima koji su značili da se radnja nije (još) dogodila.
  U modernom jeziku i dalje se koristi za negativne oblike, ali također i za pasivni i kauzativni oblik.

  Iz nje se izvode:
  \begin{itemize}
    \item ない形, ず形 --- negativni \textit{nai} i \textit{zu} oblici,
    \item 受身形 --- pasivni oblik,
  \end{itemize}
  a u klasičnom japanskom i:
  \begin{itemize}
    \item 〜ば --- kondicional.
  \end{itemize}


  \subsubsection*{\f{意志形}{い・し・けい} --- volicionalna osnova}
  Ova osnova dobiva se kod 五段 glagola pomicanjem zadnje \textit{kane} iz \textit{u}-retka u \textit{o}-redak.
  Izvedena je iz \f{未}{み}\f{然}{ぜん}\f{形}{けい} ili \textit{irrealis} osnove u modernom jeziku kao rezultat glasovne promjene kod koje \textit{a} pored \textit{u} u kombinaciji \textit{au} prelazi u \textit{o} i time nastaje produženo \textit{\=o}.
  Nalazi se samo u volicionalu (hortativu).

  \subsubsection*{\f{連用形}{れん・よう・けい} --- vezivna osnova}\label{sec:fff-----vezivna-osnova}
  Ova osnova dobiva se kod 五段 glagola pomicanjem zadnje \textit{kane} u \textit{i}-redak.

  Iz nje se izvode:
  \begin{itemize}
    \item pristojni oblici 〜ます
    \item složeni glagoli
  \end{itemize}

  \subsubsection*{\f{音便形}{おん・びん・けい} --- eufonijska osnova}
  Ova osnova izvedena je \f{連}{れん}\f{用}{よう}\f{形}{けい} ili vezivne osnove u modernom jeziku kao rezultat glasovne promjene kod koje kombinacija \textit{\textasciitilde\!~i} + \textit{t\!~\textasciitilde} kontrakcijom postaje
  \begin{center}
    \begin{tabular}{rcl}
      い/ち/り + t\textasciitilde & \textrightarrow & っt\textasciitilde \\
      び/み/に + t\textasciitilde & \textrightarrow & んd\textasciitilde \\
      き + t\textasciitilde & \textrightarrow & いt\!~\textasciitilde \\
      ぎ + t\textasciitilde & \textrightarrow & いd\textasciitilde \\
    \end{tabular}
  \end{center}


  \subsubsection*{\f{終止形}{しゅう・し・けい} --- terminativna osnova}
  Ova osnova jednaka je rječničkom obliku i završava na \textit{u}.
  Koristi se za završetke rečenica.

  \subsubsection*{\f{連体形}{れん・たい・けい} --- atributivna osnova}
  Ova osnova jednaka je rječničkom obliku i završava na \textit{u}.
  Koristi se za opis imenskih fraza.

  \subsubsection*{\f{仮定形}{か・てい・けい} --- hipotetska osnova}
  Ova osnova dobiva se kod 五段 glagola pomicanjem zadnje kane iz \textit{u}-retka u \textit{e}-redak.
  Naziva se tradicionalno još i \textit{realis} osnova, jer se u klasičnom japanskom koristila u oblicima kod kojih se radnja dogodilけい

  \subsubsection*{\f{可能形}{か・のう・けい} --- potencijalna osnova}

  \subsubsection*{\f{命令形}{めい・れい・けい} --- zapovijedna osnova}

  \subsection{Glagolski oblici}

  \subsubsection*{Imperfektiv}
  Označava generalno nesvršenu radnju ili stanje.
  Jednak je terminativnoj i atributivnoj osnovi.

  \subsubsection*{Negativ}
  Označava stanje u kojem se neka radnja ili stanje ne odvija.

  \subsubsection*{Perfektiv}
  Označava svršenu radnju ili stanje.

  \subsubsection*{\textit{\textasciitilde{}te} oblik}
  Koristi se za vezivanje i spajanje.

  \subsubsection*{Konjunktiv}
  Koristi se za vezivanje i spajanje.

  \subsubsection*{Volicional}
  Označava poziv da se učini radnja.

  \subsubsection*{Pasiv}
  Označava da subjekt prima posljedice agentove radnje.

  \subsubsection*{Kauzativ}
  Označava da je subjekt uzrokuje objekt da čini radnju.

  \subsubsection*{Imperativ}
  Označava jaku zapovijed.

  \subsubsection*{Potencijal}
  Označava moguće stanje.

  \subsubsection*{Kondicional}
  Označava uvjet u zavisnoj rečenicu koji ostvaruje posljedicu iz glav\-ne rečenice kada je ispunjen.

  \section{Čestice}

\end{document}