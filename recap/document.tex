%!TeX program = xelatex
\documentclass[a4paper, 12pt]{amsart}
\usepackage[croatian]{babel}
\usepackage[CJKchecksingle,CJKspace=true]{xeCJK}

\setCJKmainfont{Source Han Serif JP}
\setmainfont[BoldFont={* Semibold}]{Source Serif Pro}

\usepackage{indentfirst}
\usepackage{booktabs}
\usepackage{setspace}
\usepackage{enumitem}
\usepackage[defaultlines=4, all]{nowidow}

\usepackage{microtype}
%\setlist[itemize]{wide, itemsep=0.0em, topsep=0.0em, parsep=0.0em, labelwidth=!, labelindent=0.5ex, leftmargin=1ex}
%\setlist[enumerate]{wide, itemsep=0.0em, topsep=0.0em, parsep=0.0em, labelwidth=!, labelindent=0.5ex, leftmargin=1ex}

\title{Sažetak japanske gramatike za ponavljanje}
\author{Lovriša}

\onehalfspacing

\begin{document}
	\maketitle
	
	\section{Uvod}
	
	\section{Vrste riječi}
	\begin{itemize}
		\item 名詞「めいし」 --- imenice
		\item 形容詞「けいようし」 --- pridjevi
		\item 動詞「どうし」 --- glagoli
		\item 副詞「ふくし」 --- prilozi
		\item 助詞「じょし」 --- čestice
	\end{itemize}
  \subsection{名詞「めいし」 --- imenice}

  \section{Predikati i opisnice}
  \begin{itemize}
  	\item 
  \end{itemize}

  \section{Konjugirane osnove}
  \begin{itemize}
  	\item 未然形「みぜんけい」 --- \textit{a}-oblik 
  	\item 意志形「いしけい」 --- \textit{o}-oblik, volicionalni oblik
  	\item 連用形「れんようけい」 --- \textit{i}-oblik
  	\item 終止形「しゅうしけい」 --- \textit{u}-oblik, završni oblik
  	\item 連体形「れんたいけい」 --- opisni oblik
  	\item 仮定形「かていけい」 --- \textit{e}-oblik, hipotetski oblik
  	\item 可能型「かのうけい」 --- potencijalni oblik
  	\item 命令形「めいれいけい」 --- zapovijedni oblik
  	\item 音便形「おんびんけい」 --- eufonijski oblik
  \end{itemize}
  \subsection{未然形「みぜんけい」 --- \textit{a}-oblik}
  Iz njega se izvode:
  \begin{itemize}
  	\item ない形 --- negativni oblik,
    \item 受身形 --- pasivni oblik
  \end{itemize}

  \section{Glagoli}
  \section{Čestice}
  
\end{document}