\documentclass[a5paper,10pt]{tekst}

\begin{document}
	\thispagestyle{empty}
	\onehalfspacing
	
	\sffamily
	\begin{center}
		\Huge \f{都立}{とりつ} \f{高校}{こうこう}の 40% \f{髪}{かみ}が \f{茶色}{ちゃいろ}の \f{生徒}{せいと}に \f{証明書}{しょうめいしょ}を 出す ように 言う
	\end{center}
	\vspace{2em}
	
	{\Large\sloppy
		\p{東京の}\p{{都立}{高校}の}\p{40%}\p{\f{以上}{いじょう}が、}\p{髪の\f{毛}{け}が}\p{茶色や、}\p{まっすぐ}\p{では}\p{ない}\p{{生徒}に}\p{証明書を}\p{出す}\p{ように}\p{言って}\p{いる}\p{ことが}\p{わかりました。}
		\p{東京都}\p{\f{議会}{ぎかい}の}\p{\f{共産党}{きょうさんとう}の}\p{\f{議員}{ぎいん}が}\p{\f{調}{しら}べて}\p{わかりました。}
		
		\p{証明書}\p{では、}\p{生まれた}\p{ときから}\p{髪の毛が}\p{茶色や、}\p{まっすぐ}\p{では}\p{ない}\p{ことを}\p{書いて、}\p{\f{家族}{かぞく}が}\p{サイン}\p{などを}\p{します。}
		\p{小さい}\p{ときの}\p{\f{写真}{しゃしん}を}\p{出す}\p{ように}\p{言って}\p{いる}\p{\f{学校}{がっこう}も}\p{ありました。}
		
		\p{東京都の}\p{\f{教育}{きょういく}}\p{\f{委員会}{いいんかい}は}\p{「髪を}\p{\f{染}{そ}めたと}\p{先生が}\p{\f{間}{ま}\f{違}{ちが}えない}\p{ように}\p{出して}\p{もらって}\p{いますが、}\p{\f{必}{かなら}ず}\p{では}\p{ありません。}
		\p{生徒や}\p{家族}\p{などの}\p{\f{意見}{いけん}も}\p{聞いて、}\p{\f{毎年}{まいとし}、}\p{いい}\p{やり\f{方}{かた}を}\p{\f{考}{かんが}える}\p{\f{必要}{ひつよう}が}\p{あります」と}\p{\f{説明}{せつめい}}\p{して}\p{います。}
		\p{しかし、}\p{\f{専門家}{せんもんか}は}\p{「この}\p{やり方は}\p{\f{時代}{じだい}に}\p{\f{合}{あ}って}\p{いないし、}\p{子どもたちの}\p{\f{人権}{じんけん}の}\p{\f{問題}{もんだい}も}\p{あります」と}\p{\f{話}{はな}して}\p{います。}
		
	}
	
	\clearpage
	\rmfamily
	
	\begin{multicols}{2}[\subsection*{Vokabular}]
		\dictentry{都立高校 \strut}{とりつこうこう}{\item gradska gimnazija}{imenica}
		\dictentry{髪}{かみ}{\item kosa}{imenica} 
		\dictentry{茶色}{ちゃいろ}{\item smeđa}{no-pridjev, imenica} 
		\dictentry{生徒}{せいと}{\item učenik}{imenica} 
		\dictentry{証明書 \strut}{しょうめいしょ}{\item potvrda}{imenica} 
		\dictentry{以上}{いじょう}{\item više od, \textasciitilde\ i više}{priložna imenica} 
		\dictentry{髪の毛}{かみのけ}{\item kosa (dosl. dlaka na glavi)}{imenica, izraz} 
		\dictentry{東京都議会 \strut}{とうきょうとぎかい}{\item Tokijska gradska skupština}{imenica} 
		\dictentry{共産党 \strut}{きょうさんとう}{\item komunistička partija}{imenica} 
		\dictentry{議員 \strut}{ぎいん}{\item član stranke, odbora}{imenica} 
		\dictentry{調べる}{しらべる}{\item istražiti}{glagol} 
		\dictentry{生まれる}{うまれる}{\item roditi se}{glagol} 
		\dictentry{書く}{かく}{\item pisati}{glagol} 
		\dictentry{家族}{かぞく}{\item obitelj}{imenica} 
		\dictentry{小さい}{ちいさい}{\item malen}{i-pridjev} 
		\dictentry{写真}{しゃしん}{\item slika}{imenica} 
		\dictentry{学校}{がっこう}{\item škola}{imenica} 
		\dictentry{東京都 \strut}{とうきょうと}{\item prefektura Tokyo}{imenica} 
		\dictentry{教育委員会 \strut}{きょういくいいんかい}{\item odbor za obrazovanje}{imenica} 
		\dictentry{染める \strut}{そめる}{\item obojati}{glagol} 
		\dictentry{先生}{せんせい}{\item profesor}{imenica} 
		\dictentry{間違える}{まちがえる}{\item pogriješiti}{glagol} 
		\dictentry{必ず}{かならず}{\item sigurno}{prilog} 
		\dictentry{意見}{いけん}{\item mišljenje}{imenica} 
		\dictentry{毎年}{まいとし}{\item svake godine}{vremenska imenica} 
		\dictentry{聞く}{きく}{\item čuti}{glagol} 
		\dictentry{やり方}{やりかた}{\item način (na koji se nešto radi)}{imenica} 
		\dictentry{考える}{かんがえる}{\item misliti}{glagol} 
		\dictentry{必要}{ひつよう}{\item potreba, potrebno}{imenica, no-pridjev} 
		\dictentry{説明}{せつめい}{\item objašnjenje}{imenica} 
		\dictentry{専門家 \strut}{せんもんか}{\item stručnjak}{imenica}
		\dictentry{時代}{じだい}{\item period, razdoblje}{imenica, vremenska imenica}
		\dictentry{合う}{あう}{\item pristajati, biti prikladan}{glagol}
		\dictentry{子供}{こども}{\item dijete}{imenica} 
		\dictentry{人権 \strut}{じんけん}{\item ljudska prava}{imenica, no-pridjev}
		\dictentry{問題}{もんだい}{\item problem}{imenica} 
		\dictentry{話す}{はなす}{\item pričati}{glagol} 
	\end{multicols}
	
	\subsection*{Zadaci}
	\begin{enumerate}
		\item Sažmite tekst u najviše dvije rečenice.
		\item Razgovarajte o tekstu.
	\end{enumerate}
	
	\subsection*{Domaća zadaća}
	\begin{enumerate}
		\item Napišite kratku priču ili par rečenica koristeći barem 5 riječi iz teksta. \\
		Rečenice ili tekst ne moraju nužno biti vezane uz samu vijest. 
		\item Odgovorite na sljedeća pitanja
		\begin{enumerate}
			\item 生徒は何をしなければならない?
			\item 証明書とともに生徒はどのような\f{証拠}{しょうこ}を\f{渡}{わた}せばならないのですか?
			\item \sloppy\p{\f{本文}{ほんもん}の}\p{中に}\p{髪が}\p{茶色の}\p{生徒に}\p{ついて}\p{二つの}\p{意見が}\p{あります、}\p{その}\p{二つの}\p{意見を}\p{自分の}\p{で}\p{説明}\p{して}\p{ください。}
		\end{enumerate}
		\item Nadopunite sljedeće rečenice riječima iz vokabulara:
		\begin{enumerate}
			\item 花子ちゃんは\ansline{}ず自分の髪を\ansline{}めないと決めました。
			\item おばあさんは三つ\ansline{}の\f{携帯}{けいたい}のボタンを\ansline{}えた。
			\item 写真の\f{撮}{と}り方を\ansline{}しなければ\ansline{}になる\f{可能性}{かのうせい}があります。
			\item\sloppy\p{動物の}\p{\f{癖}{くせ}を}\p{\ansline{}べたいと}\p{思った}\p{鈴木さんは}\p{見つけ出した}\p{ことを}\p{ノートに}\p{\ansline{}く}\p{ことに}\p{しました。}
			\item\p{私は}\p{こう}\p{\ansline{}えた}\p{「他の}\p{人の}\p{\ansline{}}\p{なんて}\p{知らない、}\p{そんな}\p{こと}\p{\ansline{}ない」、}\p{だけど}\p{今は}\p{そう}\p{思わない、}\p{\ansline{}}\p{自体が}\p{ばか}\p{じゃないと}\p{知った}\p{から。}
		\end{enumerate}
	\end{enumerate}
\end{document}