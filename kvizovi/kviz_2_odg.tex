\documentclass[basic]{grampig}
\usepackage{graphicx}

\begin{document}
  {\Large \e {Gramatički kviz} \hfill 第二回} \vspace{1em}
  
  Upotrijebite neku od padežnih čestica u rečenicama ispod, ili ostavite prazno tako da dobijete gramatički ispravnu rečenicu.\\
  Neke čestice se ne moraju iskoristiti, a neke se ponavljaju.
  \choicebox{
    が ・ の ・ を ・ に ・ へ \br
    と ・ で ・ から ・ より ・ \resizebox{1em}{!}{$\times$}
  }
  \begin{enumerate}
    \item \parbox[t]{\linewidth}{\p{\f{僕}{ぼく}\e{の}}\p{\f{同居人}{どうきょにん}は、}\p{デカい}\p{\f{人間}{にんげん}\e{が}}\p{\f{2人}{ふたり}と、}\p{\f{茶色}{ちゃいろ}\e{の}}\p{\f{猫}{ねこ}\e{が}}\p{\f{1匹}{いっぴき}}\p{だ。}\parbox[t]{\linewidth}} \bh
    Moji cimeri su 2 ogromna ljudska bića i jedna smeđa mačka.\strut
    
    \item \p{\f{茶色}{ちゃいろ}\e{の}}\p{\f{猫}{ねこ}は}\p{モカ}\p{だ。} \bh
    Smeđa mačka se zove Moka.\strut
    
    \item \p{モカは}\p{\f{僕}{ぼく}\e{より}、}\p{かなり}\p{\f{大}{おお}きな}\p{\f{猫}{ねこ}}\p{だ。} \bh
    Moka je od mene poprilično veća mačka.\strut
    
    \item \p{この}\p{\f{家}{いえ}\e{に}}\p{\f{来}{き}た}\p{ときは、}\p{\f{僕}{ぼく}と}\p{モカは}\p{\f{同}{おな}じ}\p{\f{大}{おお}きさ}\p{だった。} \bh
    Kad smo došli u ovu kuću, ja i Moka bili smo iste veličine.\strut
    
    \item \p{けれど、}\p{この}\p{\f{3年}{さんねん}\e{の}}\p{\f{間}{あいだ}\e{に}、}\p{モカは}\p{\f{僕}{ぼく}\e{の}}\p{\f{2倍}{にばい}に}\p{なった。} \bh
    Ipak, tijekom ove tri godine, Moka je postala kao dva ja.\strut
  \end{enumerate}
\end{document}