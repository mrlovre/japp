\documentclass[intermediate]{grampig}

\begin{document}
	\begin{minipage}{\width}
		\e{\large<A>は<B>が…} \hfill struktura \br
		Struktura kojom se tvrdnja o <B> vezuje uz <A>, gdje <A> nema izravnu kontrolu nad <B>. \\
		Koristi se kod tvrdnji koji se tiču želje, volje, mogućnosti, posjedovanja i emocija, i sl.
		
		\begin{table}
			\centering
			%			\renewcommand{\fsep}{-0.1em}
			\begin{tabular}{@{}ccccccc@{}}
				%				<A> & \e{は} & <B> & \e{が} & … \\
				%				\f{彼}{かれ} & & \f{背}{せ} & & 高い \\
				\f{彼}{かれ} & \e{は} & \f{背}{せ} & \e{が} & 高い & $\Rightarrow$ & \f{彼}{かれ}は\f{背}{せ}が高い \bh
				on & & stas & & visok & & njemu je stas visok \br
				私 & \e{は} & お\f{金}{かね} & \e{が} & ない & $\Rightarrow$ & 私はお\f{金}{かね}がない \bh
				ja & & novac & & nema & & meni nema novca
			\end{tabular}
		\end{table}
		
		\begin{itemize}
			\item \f{彼女}{かのじょ}\e{は}そんな\f{音楽}{おんがく}\e{が}\f{嫌}{きら}いだ。\bh
			Njoj je takva glazba mrska.
			\item 私\e{は}日本の\f{文化}{ぶんか}に\f{興味}{きょうみ}\e{が}ある。\bh
			Meni/za mene je japanska kultura zanimljiva.
			\item \f{美奈子}{みなこ}さん\e{は}先生が聞いた\f{質問}{しつもん}\e{が}分からなかった。\bh
			(Za) Minako pitanje koje je učitelj pitao nije bilo razumljivo.
			%				\item \f{敏郎}{としろう}\e{は}クモ\e{が}\f{怖}{こわ}い。\bh
			%				Za Toshir\={o}a/Toshir\={o}u su pauci strašni.
		\end{itemize}
	\end{minipage}
\end{document}
