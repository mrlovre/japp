\documentclass[intermediate]{grampig}

\begin{document}
	\begin{minipage}{\width}
		\e{\large だけ} \hfill čestica / zavisna imenica \br
		Čestica koja znači \textit{isključivo}, \textit{jedino} ili \textit{samo}.
		Kao zavisna imenica govori koliko (malo) je nečega.
		
		\begin{table}
			\centering
			\begin{tabular}{@{}ccccc@{}}
				一度 & + & \e{だけ} & $\Rightarrow$ & いちどだけ \bh
				jednom & & & & samo jednom \br
				できる & + & \e{だけ} & $\Rightarrow$ & できるだけ \bh
				moći & & & & (toliko) koliko možeš
			\end{tabular}
		\end{table}
		
		\begin{itemize}
			\item \ruby{健二郎}{けんじろう}さんは\ruby{今夜}{こんや}ビールを\e{一杯だけ}飲んだ。\bh
			Kenjir\={o} je večeras popio \e{samo jedno} pivo.
			\item \ruby{北}{きた}\ruby{朝鮮}{ちょうせん}には\e{\ruby{金}{キム}\ruby{正恩}{ジョンウン}さまだけ}\ruby{一日}{いちにち}\ruby{三回}{さんかい}食べる。\bh
			U Sjevernoj Koreji \e{jedino Kim Jong-Un} tri puta dnevno jede.
			\item \e{\ruby{男子}{だんし}\ruby{学生}{がくせい}だけが}\ruby{工学}{こうがく}を勉強しているというルールはもうない。\bh
			Više nema pravila da \e{samo muški studenti} uče inženjerstvo.			
		\end{itemize}
	\end{minipage}
\end{document}
