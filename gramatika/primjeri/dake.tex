\documentclass[intermediate]{grampig}

\begin{document}
	\begin{minipage}{\width}
		\e{\large だけ} \hfill priložna čestica \br
		Čestica koja znači \e{isključivo}, \e{jedino} ili \e{samo}, gdje se ističe koliko (malo) je nečega.
		S potencijalnim oblikom glagola također odgovara na pitanje \e{koliko}.
		
		\vspace{-0.5em}
		\begin{table}
			\centering
			\begin{tabular}{@{}ccccc@{}}
				一度 & + & \e{だけ} & $\Rightarrow$ & いちどだけ \bh
				jednom & & & & samo jednom \br
				できる & + & \e{だけ} & $\Rightarrow$ & できるだけ \bh
				moći & & & & koliko možeš
			\end{tabular}
		\end{table}
		\vspace{-0.5em}
		
		\begin{itemize}
		\item \f{健二郎}{けんじろう}さんは\f{今夜}{こんや}ビールを\e{\f{一杯}{いっぱい}だけ}飲んだ。\bh
		Kenjir\={o} je večeras popio \e{samo jedno} pivo.
		
		\item \f{北朝鮮}{きたちょうせん}には\e{\f{金正恩}{キム・ジョン・ウン}さまだけ}\f{一日}{いちにち}\f{三回}{さんかい}食べる。\bh
		U Sjevernoj Koreji \e{jedino Kim Jong-Un} tri puta dnevno jede.
		
		% \item 今日の仕事は\e{レポートを書くだけ}だった。\f{}{\strut}\bh
		% Današnji posao je bio \e{samo pisanje izvještaja}.
		
		% \item \e{\f{男子}{だんし}\f{学生}{がくせい}だけ}\f{工学}{こうがく}を勉強しているというルールはもうない。\bh
		% Više nema pravila da \e{samo muški studenti} uče inženjerstvo.
		
		
		\end{itemize}
	\end{minipage}
\end{document}
