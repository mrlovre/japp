\documentclass[pig]{grampig}
\usepackage{ragged2e}
\usepackage{setspace}
\usepackage{makecell}

\newcolumntype{L}{l}
\newcolumntype{R}{>{\color{\accentcolor}}r}
\newcolumntype{C}{>{\itshape}c}

\begin{document}
	\begin{minipage}{\width}
		\onehalfspacing
%		{\Large \e{PIG Turnir}}  \\[-0.5em]
%		
%		Novost na Makoto tržištu! 
%		Ne dva, ne tri, već četiri u jedan, a možda čak i više! Sve ovisi o Vama!  \\[-0.5em]
%		
%		Ogledajte se u sklapanju rečenice, jedne ili više njih, možda čak i sastava!
%		Jedino pravilo jest da upotrijebite sve tri riječi, \e{pridjev-imenicu-glagol}, ne nužno u gramatičkom formatu u kojem su napisane: negirajte ih, bacajte u prošlost, radite im što god Vas volja. 
%		Riječ na katakani dobivate gratis, možete je, ali i ne morate upotrijebiti.  \\[-0.5em]
%		
%		Vi samo pišite --- naše je da čitamo.
%		Vi samo pitajte --- naše je da odgovaramo. 
%		Bojite se grešaka? Ne bojte se!
%		Tko radi, taj i griješi. 
%		A tko griješi, taj i uči. \\[-1.5em]
		
		\begin{table}
			\centering
			\textsc{\e{Primjer:}} \br
			\begin{tabular}{RCl}
				青い & i-pridjev & plavo \\
				\ruby{空}{そら} & imenica & nebo \\
				\ruby{焼}{や}く & u-glagol & peći \\
				\f{ハンバーガー}{\strut} & katakana & hamburger \\
			\end{tabular}
		\end{table} \vspace{-1em}
		
		\singlespacing
		\begin{itemize}
			\item 明日は\e{青い空}の下で\e{ハンバーガー}を\e{焼きます}。\bh
			Sutra ću ispod \e{plavog neba} \e{peći hamburgere}. \\[-0.5em]
			
			\item \e{空}が\e{青くて}、パンを\e{焼いています}。\bh
			\e{Nebo} je \e{plavo}, \e{pečem} kruh. \\[-0.5em]
			
			\item \e{青くない}\f{電車}{でんしゃ}に乗っていて、\f{窓}{まど}から\e{空}を見ていた。\\
			家に帰ったら、ソーセージを\e{焼いた}。\bh
			Vozio sam se u \e{neplavom} vlaku i gledao \e{nebo}. \\
			Kad sam došao kući, \e{ispekao sam} kobasice. 
		\end{itemize} \vspace{-1.5em}
	
		\begin{center}
			\e{今から、\f{PIG}{ぴっぐ}しよう!}
		\end{center}
	\end{minipage}
\end{document}