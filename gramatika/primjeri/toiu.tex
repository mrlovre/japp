\documentclass[intermediate]{grampig}

\begin{document}
	\begin{minipage}{\width}
		\e{\large 〜という} \hfill opisna rečenica \br
		Zavisna rečenica koja opisuje citirajući punu rečenicu. \\
		Prevodi se kao \textit{koji se zove}, \textit{koji kaže}, ili kao neupravni govor pomoću riječi \textit{da}.
		
		\begin{table}
			\centering
			\begin{tabular}{@{}ccccc@{}}
				デュラララ & + & \e{という} & $\Rightarrow$ & デュラララという(\f{小説}{しょうせつ}) \bh
				Durarara & & & & (roman) zvani/koji se zove Durarara \br
				\f{授業}{じゅぎょう}がない & + & \e{という} & $\Rightarrow$ & \f{授業}{じゅぎょう}がないという(\f{知}{し}らせ) \bh
				nema nastave & & & & (obavijest) da nema nastave
			\end{tabular}
		\end{table}
		
		\begin{itemize}
			\item \e{\f{何}{なん}という}\f{意味}{いみ}ですか? \bh
			Što znači? (\e{Što kaže} značenje ovog?)
		\end{itemize}
	\end{minipage}
\end{document}
