\documentclass[basic]{grampig}

\begin{document}
	\begin{minipage}{\width}
		\e{\large 〜として} \hfill složena čestica \br
		Objašnjava \textit{u svojstvu koga/čega} netko/nešto radi.
		
		\begin{table}
			\centering
			\begin{tabular}{@{}ccccc@{}}
				\f{女優}{じょゆう} & + & \e{として} & $\Rightarrow$ & \f{女優}{じょゆう}として \bh
				glumica & & & & kao glumica
			\end{tabular}
		\end{table}
		
		\begin{itemize}
			\item ここには\e{学生として}働いている。\bh
			Ovdje radim \e{kao student}.
			\item ドーナツは\e{お\f{菓子}{かし}として}\f{甘}{あま}く、\f{砂糖}{さとう}を上に\f{散}{ち}らす\f{必要}{ひつよう}がない。\bh
			Krafna je \e{kao poslastica} slatka, ne treba šećer po njoj posipati.
			\item 〈\e{\f{田中}{たなか}\f{太郎}{たろう}として}\f{続}{つづ}く〉 \bh
			<Nastaviti \e{kao Pero Perić}> (čest izraz u računalnom svijetu)
		\end{itemize}
	\end{minipage}
\end{document}
