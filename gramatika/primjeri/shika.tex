\documentclass[basic]{grampig}

\begin{document}
	\begin{minipage}{\width}
		\e{\large しか} \hfill čestica \br
		Uz negativan predikat znači \e{doli}, \e{ništa osim} ili \e{nego (samo)}.
		
		\begin{table}
			\centering
			\begin{tabular}{@{}ccccc@{}}
				私 & + & \e{しか} & $\Rightarrow$ & 私しか \bh
				ja & & & & (nitko) osim mene \br
%				行く & + & \e{しか} & $\Rightarrow$ & 行くしか(ない) \bh
%				ići & & & & (nema) nego ići
			\end{tabular}
		\end{table}
		
		\begin{itemize}
			\item 私は\f{弟}{おとうと}が\e{一人しか}いない。\bh
			Nemam \e{nego (samo) jednog} brata.
			\item 内の猫は\e{\f{一番}{いちばん}高い食べ物しか}食べません。\bh
			Naš mačak ne jede \e{doli najskuplju hranu}.
			\item \f{奈良}{なら}にいるなら、\e{\f{鹿}{しか}しか}見えない。\bh
			Ako si u Nari, nećeš vidjeti \e{ničeg osim jelena}.
		\end{itemize}
	\end{minipage}
\end{document}
