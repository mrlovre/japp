\documentclass[advanced]{grampig}

\begin{document}
	\begin{minipage}{\width}
		\e{\large 〜の} \hfill zavisna imenica \br
		Zavisna neodređena zamjenica. \\
		Na hrvatski se prevodi opcionalno nekom pokaznom zamjenicom (npr. \textit{onaj}), a na engleskom s \textit{one} (npr. \textit{the big one}).
		
		\begin{table}
			\centering
			\begin{tabular}{@{}ccccc@{}}
				大きい & + & \e{の} & $\Rightarrow$ & 大きいの \bh
				velik & & & & (onaj) veliki \br
				先生が言った & + & \e{の} & $\Rightarrow$ & 先生が言ったの \bh
				učitelj je rekao & & & & (ono) što je učitelj rekao
			\end{tabular}
		\end{table}
		\begin{itemize}
			\item \f{黒}{くろ}い\f{子猫}{こねこ}は\f{箱}{はこ}の\f{後}{うし}ろで\f{寝}{ね}ていて、\e{\f{白}{しろ}いの}は\f{外}{そと}で\f{遊}{あそ}んでいる。\bh
			Crni mačić spava iza kutije, a \e{(onaj) bijeli} se igra vani.
			\item \f{写真}{しゃしん}は三四\f{枚}{まい}を\f{撮}{と}って、後で\e{\f{一番}{いちばん}きれいなの}を\f{投稿}{とうこう}しなさい。\bh
			Uslikaj 3--4 fotografije, pa poslije objavi \e{(onu) najljepšu}.
			\item \e{昨日先生が言ったの}を\f{覚}{おぼ}えている?\bh
			Sjećaš li se \e{(onog) što je jučer učitelj rekao}?
		\end{itemize}
	\end{minipage}
\end{document}
