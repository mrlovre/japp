\documentclass[intermediate,bless]{grampig}

\title{〜の}
\pos{zamjenica}

\begin{document}
	\maketitle
	Zamjenjuje imenicu koja dolazi iza nekog opisa. \\
%	Na hrvatskom prijevodu se uz opis može pojaviti i pokazna zamjenica (npr. onaj veliki).
	Može se prevesti kao neka pokazna zamjenica (npr. onaj) ili samo opisom.
%	Na engleskom se prevodi kao \textit{one}.
	
	\begin{table}
		\centering
		\begin{tabular}{@{}ccccc@{}}
			青い & + & \e{の} & $\Rightarrow$ & 青いの \bh
			plavi & & & & (onaj koji je) plavi \\
%			& & & & (\textit{the blue one})
%			見た & + & \e{の} & $\Rightarrow$ & 見たたの \bh
%			vidio sam & & & & (ono) što sam vidio
		\end{tabular}
	\end{table}
	\begin{itemize}
		\item \f{黒}{くろ}い\f{子猫}{こねこ}は\f{箱}{はこ}の\f{後}{うし}ろで\f{寝}{ね}ていて、\e{\f{白}{しろ}いの}は\f{外}{そと}で\f{遊}{あそ}んでいる。\bh
		Crni mačić spava iza kutije, a \e{bijeli} se igra vani.
		
		\item \f{写真}{しゃしん}は三四\f{枚}{まい}を\f{撮}{と}って、後で\e{\f{一番}{いちばん}きれいなの}を\f{投稿}{とうこう}しなさい。\bh
		Uslikaj 3--4 fotografije, pa poslije objavi \e{(onu) najljepšu}.
		
		\item \e{昨日先生が言ったの}を\f{覚}{おぼ}えている?\bh
		Sjećaš li se \e{(onog) što je jučer učitelj rekao}?
	\end{itemize}
\end{document}
