\documentclass[ocha]{grampig}
\usepackage{ragged2e}
\usepackage{setspace}

\newcolumntype{L}{l}
\newcolumntype{R}{>{\color{\accentcolor}}r}
\newcolumntype{C}{>{\itshape}c}

\begin{document}
	\begin{minipage}{\width}
		\onehalfspacing
		{\Large \e{Tematski O-Čaj}} \\[-0.5em]
		
		Nadaleko poznati Makoto bar, \e{Popodnevni O-Čaj}, poznat po neformalnim druženjima i prevođenju osebujnih sadržaja s japanskog na hrvatski i obrnuto, sada otvara novu poslovnicu nedaleko od Vas! \\[-1em]
		
		\e{Tematski O-čaj} moderirana je, poluformalna verzija Makoto bara.
		Kavu / čaj / kekse si pripremate sami, a te\-me po\-slu\-žu\-je\-mo mi.
		Ali, samo na početku, jer čim kročite u \e{O-Čaj}, postajete dio očaja\ldots\ pravog, istinskog, japanskog!
		Stoga se ne libite iznijeti na stol i Vaše vlastite tematske specijalitete!
		Rado ćemo se podružiti i uz njih! \\[-1em]
		
		Prvo \e{O-Čajavanje} nosi varljivo kratak naslov \e{,,Japanski i ja''}. 
		Zašto ste počeli učiti japanski?
		Što Vas je privuklo? Koja je Vaša slaba japanska točka koja je slomila otpor prema čak trima vrstama pisma?
		Podijelite Vaša mišljenja i doživljaje s ostalima! \\[-2em]
		
		\begin{center}\e{
				U ovu srijedu, 17.3. u 19.00 sati na Discordu. \\
				Svi su pozvani i dobrodošli!
			}
		\end{center}
	\end{minipage}
\end{document}