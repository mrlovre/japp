\documentclass[intermediate]{grampig}

\begin{document}
	\begin{minipage}{\width}
		\e{\large よう(に)} \hfill zavisna imenica/prilog \br
		%		Zavisna imenica čiji opis objašnjava \textit{kako} se nešto događa. \\
		%		Uz imenicu X znači \textit{na način na koji X radi}.
		Zavisna imenica koja čiji opis govori kako se nešto \textit{čini} ili \textit{izgleda}. \\
		U obliku priloga (s に) govori \textit{na koji način} se nešto radi ili događa.
		
		\begin{table}
			\centering
			\begin{tabular}{@{}ccccc@{}}
				日本人 & + & \e{のよう} & $\Rightarrow$ & 日本人のよう \bh
				japanci & & & & kao japanci \br
				食べる & + & \e{よう} & $\Rightarrow$ & 食べるよう \bh
				jesti & & & & kao da jede
			\end{tabular}
		\end{table}
		
		\begin{itemize}
			\item \f{宮本}{みやもと}さんのカバンは本でいっぱいで、\e{\f{重}{おも}いようです}。\bh
			Mijamotova torba je puna knjiga, \e{čini se teška}.
			% \item \f{胸}{むね}が\e{\f{青眼の白龍}{ブルーアイズ・ホワイト・ドラゴン}をたった今見たように}\f{打}{う}っています。\bh
			\item \f{胸}{むね}が\e{\f{亡}{な}きチトー\f{元帥}{げんすい}の\f{幽霊}{ゆうれい}を見たように}\f{打}{う}っています。\bh
			Srce mi lupa \e{kao da sam vidio duha pokojnog maršala Tita}.
			\item \e{読めるように}書いてください。\bh
			Molim te piši \e{tako da se može pročitati}.
		\end{itemize}
	\end{minipage}
\end{document}
