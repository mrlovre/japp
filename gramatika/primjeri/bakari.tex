\documentclass[intermediate]{grampig}

\begin{document}
	\begin{minipage}{\width}
		\e{\large ばかり} \hfill priložna čestica \br
		Čestica koja znači \e{tek}, \e{isključivo}, \e{jedino} ili \e{s\=ami}, gdje se ističe vrsta nečega, ali ne i količina. \\
		%		Koristi se samo s pozitivnim oblicima. 
		Može se prevesti i kao \e{samo}, ali tada značenje postaje dvosmisleno.
		Na engleskom je to \textit{\e{just}}, a ne \textit{\e{only}}.
		
		\begin{table}
			\centering
			\begin{tabular}{@{}ccccc@{}}
				本 & + & \e{ばかり} & $\Rightarrow$ & 本ばかり \bh
				knjige & & & & s\=ame knjige \br
				来た & + & \e{ばかり} & $\Rightarrow$ & 来たばかり \bh
				došao & & & & tek došao
			\end{tabular}
		\end{table}
		
		\begin{itemize}
			% 男の子は\f{一日中}{いちにちじゅう}\f{宿題}{しゅくだい}\e{ばかり}をしています。\bh
			% Dječak cijeli dan radi \e{isključivo zadaću}. \br
			\item \f{健二郎}{けんじろう}さんは\f{今夜}{こんや}お酒\e{ばかり}を飲んでいます。\bh
			Kenjir\={o} večeras pije \e{s\=amu rakiju}.
			
			\item 家に\e{\f{帰}{かえ}ったばかり}で、もうすぐ仕事に行きます。\bh
			\e{Tek sam se vratio} kući, a uskoro već idem na posao.
			
			\item 今日の仕事は\e{レポートを書くばかり}だった。\f{}{\strut}\bh
			Današnji posao je bio \e{isključivo pisanje izvještaja}.
		\end{itemize}
	\end{minipage}
\end{document}
