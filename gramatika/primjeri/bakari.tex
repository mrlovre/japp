\documentclass[intermediate]{grampig}

\begin{document}
	\begin{minipage}{\width}
		\e{\large ばかり} \hfill čestica / zavisna imenica \br
		Čestica koja znači \textit{tek}, \textit{isključivo}, \textit{jedino} ili \textit{(sve) s\=ami}, gdje se ograničava vrsta nečega, ali ne i količina. \\
		Koristi se samo s pozitivnim oblicima. Može se prevesti i sa \textit{samo}, ali tada postaje dvosmisleno.
		Na engleskom je to \textit{just}, a ne \textit{only}.
		
		\begin{table}
			\centering
			\begin{tabular}{@{}ccccc@{}}
				本 & + & \e{ばかり} & $\Rightarrow$ & 本ばかり \bh
				knjige & & & & samo knjige \br
				来た & + & \e{ばかり} & $\Rightarrow$ & 来たばかり \bh
				došao & & & & tek došao
			\end{tabular}
		\end{table}
		
		\begin{itemize}
%			\item 男の子は\f{一日中}{いちにちじゅう}\f{宿題}{しゅくだい}\e{ばかり}をしています。\bh
%			Dječak cijeli dan radi \e{isključivo zadaću}.
			\item \f{健二郎}{けんじろう}さんは\f{今夜}{こんや}お酒\e{ばかり}を飲んでいます。\bh
			Kenjir\={o} večeras pije \e{s\=amu rakiju}.
			\item 家に\e{\f{帰}{かえ}ったばかり}で、もうすぐ仕事に行きます。\bh
			\e{Tek sam se vratio} kući, a uskoro već idem na posao.
			\item 今日の仕事は\e{レポートを書くばかり}だった。\f{}{\strut}\bh
			Današnji posao je bio \e{(sve) s\=amo pisanje izvještaja}.
		\end{itemize}
	\end{minipage}
\end{document}
