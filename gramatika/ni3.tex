\documentclass[basic]{grampig}

\title{に\textsuperscript{3}}
\pos{čestica}

\begin{document}
  \begin{minipage}{\width}
    \maketitle
    Označava mjesto gdje neko stanje traje. \\
    Odgovara na pitanje \e{gdje}. \\
    Ne može se koristiti za glagole radnje.

    \vspace{0.5em}

    \begin{table}
      \centering
      \begin{tabular}{@{}ccccc@{}}
        \f{母}{はは} & + & に & $\Rightarrow$ & \f{母}{はは}に \bh
        \textit{majka} & & & & \textit{majci} \br
        \f{図書館}{としょかん} & + & に & $\Rightarrow$ & \f{図書館}{としょかん}に \bh
        \textit{knjižnica} & & & & \textit{u knjižnicu}
      \end{tabular}
    \end{table}

    \vspace{0.5em}

    \begin{itemize}
      \item \f{手紙}{てがみ}を\e{\f{母}{はは}に}\f{送}{おく}れる。\bh
      \textit{Poslat ću pismo \e{majci}.}
      \item \e{コンビニの\f{前}{まえ}に}\f{友達}{ともだち}を\f{待}{ま}っている。\bh
      \textit{\e{Ispred trgovine} čekam prijatelja.}
      \item 小さな子猫は\e{\f{図書館}{としょかん}に}\f{入}{はい}っていた。\bh
      \textit{Mali mačić ušao je \e{u knjižnicu}.}
    \end{itemize}
  \end{minipage}
\end{document}
