\documentclass[intermediate,bless]{grampig}

\title{だけ}
\pos{čestica}

\begin{document}
	\begin{minipage}{\width}
		\maketitle
		Ističe da osim onog što je navedeno nema ničeg drugog. \\
		Prevodimo je najčešće sa \e{samo}, \e{isključivo} ili \e{jedino}.
%		Na engleskom je to \textit{only}, a ne \textit{just}.
		
%		S potencijalnim oblikom glagola također odgovara na pitanje \textit{koliko}.
		
		\vspace{-0.5em}
		\begin{table}
			\centering
			\begin{tabular}{@{}ccccc@{}}
				りんご & + & \e{だけ} & $\Rightarrow$ & りんごだけ \bh
				jabuke & & & & \makecell[t]{samo jabuke \bh (i ništa više)} \br
				\f{一度}{いちど} & + & \e{だけ} & $\Rightarrow$ & \f{一度}{いちど}だけ \bh
				jednom & & & & samo jednom \br
%				できる & + & \e{だけ} & $\Rightarrow$ & できるだけ \bh
%				moći & & & & koliko možeš
			\end{tabular}
		\end{table}
		\vspace{-0.5em}
		
		\begin{itemize}
		\item \f{今夜}{こんや}\f{健二郎}{けんじろう}さんはビールを\e{\f{一杯}{いっぱい}だけ}飲んだ。\bh
		Večeras je Kenjir\={o} popio \e{samo jedno} pivo.
		
		\item \f{北朝鮮}{きたちょうせん}には\e{\f{金正恩}{キム・ジョン・ウン}さまだけ}が\f{一日}{いちにち}\f{三回}{さんかい}食べる。\bh
		U Sjevernoj Koreji \e{jedino Kim Jong-Un} tri puta dnevno jede.
		
		\item \f{荷物}{にもつ}は\e{これだけ}です。\bh
		Prtljaga je \e{samo ovo}. (Ovo je sva prtljaga.)
		
		% \item 今日の仕事は\e{レポートを書くだけ}だった。\f{}{\strut}\bh
		% Današnji posao je bio \e{samo pisanje izvještaja}.
		
%		\item インテルリーベルで\e{この\f{辞書}{じしょ}だけ}を買いました。\bh
%		Na Interliberu sam kupio \e{samo ovaj rječnik}.

%		\item あたしの人生は渡らなければならない\e{\f{橋}{はし}だけ}です。\bh
%		Život moj je \e{samo most} preko kojeg treba proć.
		
		\end{itemize}
	\end{minipage}
\end{document}
