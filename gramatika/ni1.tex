\documentclass[basic]{grampig}

\title{に\textsuperscript{1}}
\pos{čestica}

\renewcommand{\textit}{}

\begin{document}
	\begin{minipage}{\width}
		\maketitle
		Označava primatelja radnje. \\
		Odgovara na pitanja \e{komu} ili \e{čemu}.

    \vspace{0.5em}

		\begin{table}
			\centering
			\begin{tabular}{@{}ccccc@{}}
				\f{母}{はは} & + & に & $\Rightarrow$ & \f{母}{はは}に \bh
				\textit{majka} & & & & \textit{majci}
			\end{tabular}
		\end{table}

    \begin{itemize}
			\item \f{手紙}{てがみ}を\e{\f{母}{はは}に}\f{書}{か}く。\bh
			\textit{Napisat ću pismo \e{majci}.}
			\item \f{坂本}{さかもと}さんが\e{\f{猫}{ねこ}に}\f{餌}{えさ}を\f{上}{あ}げた。\bh
      \textit{Gđa.\ Sakamoto je mački dala hranu.}
			\item \f{花子}{はなこ}さんがパリーから\f{絵葉書}{えはがき}を\f{一枚}{いちまい}も\e{\f{武}{たけし}くんに}\f{送}{おこ}らなかった。\bh \textit{Hanako nije poslala ni jednu razglednicu iz Pariza \e{Takešiju}.}
    \end{itemize}
	\end{minipage}
\end{document}
